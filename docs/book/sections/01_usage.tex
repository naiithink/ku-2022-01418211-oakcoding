\section{วิธีการใช้งาน}
\label{sec:usage}
% \addcontentsline{toc}{section}{วิธีการใช้งาน \appName}

\setlength{\baselineskip}{1ex}

\begin{itemize}[leftmargin=0pt]
    \item[] \nameref{subsec:prep-install}
    \item[] \nameref{subsec:install}
    \item[] \nameref{subsec:launch-app}
\end{itemize}

\rule{0em}{1ex}

\subsection{เตรียมตัวสำหรับการเปิดใช้งาน}
\label{subsec:prep-install}
\phantomsection

ก่อนที่เราจะเริ่มต้นกัน คุณจะต้องเตรียมสิ่งต่อไปนี้:

\begin{itemize}
    \item คอมพิวเตอร์ส่วนบุคคล
    \item การเชื่อมต่ออินเทอร์เน็ต
    \item บัญชี \href{https://github.com/}{GitHub} ที่มีสิทธิ์เข้าถึง \href{https://github.com/CS211-651/project211-oakcoding.git}{Git repository ของ OakCoding}
    \item \href{https://git-scm.com/}{Git}
    \item \href{https://openjdk.org/}{Java Development Kit 17}
    \item \href{https://maven.apache.org/}{Maven}
\end{itemize}

เมื่อคุณมีครบทุกสิ่งทุกอย่างแล้ว คุณก็พร้อมที่จะ\nameref{subsec:install}

\clearpage
\subsection{ดาวน์โหลดและติดตั้ง OakCoding}
\label{subsec:install}

หลังจากที่คุณมีทุกสิ่งทุกอย่างพร้อมแล้ว หากยังโปรดไปที่\linebreak[3]\mbox{\textbf{\nameref{subsec:prep-install}}}\\

\rule{0em}{1ex}

สำหรับ macOS และระบบปฏิบัติการคล้าย UNIX อื่น ๆ คุณสามารถดาวน์โหลด\linebreak[3] \mbox{และติดตั้ง} OakCoding ได้ตามขั้นตอนดังต่อไปนี้

\begin{enumerate}
\setlength{\itemsep}{0.7pt}
    \item เปิดแอป Terminal\marginpar{สำหรับความช่วยเหลือเกี่ยวกับแอป Terminal\linebreak[4]
    โปรดไปที่ \href{https://support.apple.com/th-th/guide/terminal/welcome/mac/}{support.apple.com/th-th/guide/terminal/welcome/mac/}}

    \oakparimage{sections/images/01/01-01_terminal-app.png}

    \item จากนั้นคุณจะต้องดาวน์โหลดไฟล์ทีี่จำเป็นด้วยคำสั่ง:

\begin{lstlisting}[numbers=none]
$ git clone %\oakrepourl%
\end{lstlisting}

    หากคุณได้รับข้อความว่า \texttt{Username for 'https://github.com': } ให้คุณทำการลงชื่อเข้าใช้
    ด้วยบัญชี GitHub ของคุณ\marginpar{บัญชี GitHub ของคุณจะต้องมีสิทธิ์เข้าถึง Git repository ของ OakCoding ด้วย
    ไม่เช่นนั้นคุณจะไม่สามารถดาวน์โหลดไฟล์ต่าง ๆ ได้}

    \item เมื่อการดาวน์โหลดเสร็จสิ้น ให้คุณตรวจสอบว่ามี directory ที่ชื่อว่า \mbox{\texttt{project211-oakcoding}} อยู่หรือไม่ด้วยคำสั่ง

\begin{lstlisting}[numbers=none]
$ ls
\end{lstlisting}

    หากการดาวน์โหลดนั้นเสร็จสมบูรณ์ จะต้องมีชื่อ directory ปรากฏอยู่

\begin{lstlisting}[numbers=none]
$ ls
project211-oakcoding
\end{lstlisting}
\end{enumerate}

\clearpage

\subsection{เปิดใช้งานแอป}
\label{subsec:launch-app}

\rule{0em}{1ex}

คุณสามารถเปิดใช้งานแอปได้โดยการ

\subsection{การใช้งานสำหรับผู้ใช้ระบบในระดับสิทธิ์ต่าง ๆ}
ความสามารถของ application ที่จำเพาะต่อผู้ใช้ระบบในแต่ละระดับสิทธิ์

\subsubsection{ผู้ดูแลระบบ (Admin)}
\noindent\blindtext[3]

\subsubsection{เจ้าหน้าที่ (Staff)}
\noindent\blindtext[3]

\subsubsection{นิสิต (Consumer)}
\noindent\blindtext[3]
