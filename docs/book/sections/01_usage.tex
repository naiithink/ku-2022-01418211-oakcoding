\documentclass[../docs.tex]{subfiles}

\begin{document}
\section{วิธีการใช้งาน}
\addcontentsline{toc}{section}{วิธีการใช้งาน \appName}

% \linespread{1ex}
\setlength{\baselineskip}{1ex}


\begin{enumerate}
    \setlength{\itemsep}{0.7pt}
    \item เตรียมตัวสำหรับการเปิดใช้งาน
    \item ดาวน์โหลดและติดตั้ง OakCoding
    \item เปิดใช้งานแอป
\end{enumerate}

\rule{0em}{1ex}

\subsection*{เตรียมตัวสำหรับการเปิดใช้งาน}\label{sec:prerequisites}

ก่อนที่เราจะเริ่มต้นกัน คุณจะต้องเตรียมสิ่งต่อไปนี้:

\begin{itemize}
    \setlength{\itemsep}{0.7pt}
    \item คอมพิวเตอร์ส่วนบุคคลที่มี \href{https://www.apple.com/macos/}{macOS} ติดตั้งอยู่
    \item การเชื่อมต่อกับอินเทอร์เน็ต
    \item คอมพิวเตอร์ส่วนบุคคลของคุณจะต้องติดตั้ง software ต่อไปนี้
    \item โปรแกรม \href{https://git-scm.com/}{Git}
    \item โปรแกรม \href{https://openjdk.org/}{JDK17}
    \item โปรแกรม \href{https://maven.apache.org/}{Maven}
\end{itemize}

เมื่อคุณมีครบทุกสิ่งทุกอย่างแล้ว คุณก็พร้อมสำหรับการ\textbf{ดาวน์โหลดและติดตั้ง OakCoding}

\label{subsec:installation}
\subsection*{ดาวน์โหลดและติดตั้ง OakCoding}

หลังจากที่คุณมีทุกสิ่งทุกอย่างพร้อมแล้ว หากยังโปรดไปที่ \nameref{sec:prerequisites}\\
\rule{0em}{1ex}\\
สำหรับ macOS คุณสามารถดาวน์โหลดและติดตั้ง OakCoding ได้ตามขั้นตอนต่อไปนี้

\begin{enumerate}
\setlength{\itemsep}{0.7pt}
    \item เปิด app Terminal (Terminal.app)

\begin{lstlisting}[frame=none,numbers=none]
ls
\end{lstlisting}
\end{enumerate}

\label{launch-app}
\subsection*{เปิดใช้งานแอป}

\rule{0em}{1ex}

คุณสามารถเปิดใช้งาน app ได้โดยการ

\subsection{การใช้งานสำหรับผู้ใช้ระบบในระดับสิทธิ์ต่าง ๆ}
ความสามารถของ application ที่จำเพาะต่อผู้ใช้ระบบในแต่ละระดับสิทธิ์

\subsubsection{ผู้ดูแลระบบ (Admin)}
\noindent\blindtext[3]

\subsubsection{เจ้าหน้าที่ (Staff)}
\noindent\blindtext[3]

\subsubsection{นิสิต (Consumer)}
\noindent\blindtext[3]
\end{document}
