\section{วิธีการใช้งาน}\label{sec:usage}
% \addcontentsline{toc}{section}{วิธีการใช้งาน \appName}

\setlength{\baselineskip}{1ex}

\begin{itemize}[leftmargin=0pt]
    \item[] \nameref{subsec:prep-install}
    \item[] \nameref{subsec:install}
    \item[] \nameref{subsec:launch-app}
\end{itemize}

\rule{0em}{1ex}

\subsection{เตรียมตัวสำหรับการเปิดใช้งาน}\label{subsec:prep-install}
\phantomsection

ก่อนที่เราจะเริ่มต้นกัน คุณจะต้องเตรียมสิ่งต่อไปนี้:

\begin{itemize}
    \item คอมพิวเตอร์ส่วนบุคคล
    \item การเชื่อมต่ออินเทอร์เน็ต
    \item บัญชี \href{https://github.com/}{GitHub} ที่มีสิทธิ์เข้าถึง \href{https://github.com/CS211-651/project211-oakcoding.git}{Git repository ของ OakCoding}
    \item \href{https://git-scm.com/}{Git}
    \item \href{https://openjdk.org/}{Java Development Kit 17}
    \item \href{https://maven.apache.org/}{Maven}
\end{itemize}

เมื่อคุณมีครบทุกสิ่งทุกอย่างแล้ว คุณก็พร้อมที่จะไปที่ \oaksecref{subsec:install}

\clearpage

\subsection{ดาวน์โหลด OakCoding}\label{subsec:install}

หลังจากที่คุณมีทุกสิ่งทุกอย่างพร้อมแล้ว หากยังโปรดไปที่\\
\mbox{\oaksecref{subsec:prep-install}}

\rule{0em}{1ex}

สำหรับ macOS และระบบปฏิบัติการคล้าย UNIX อื่น ๆ คุณสามารถดาวน์โหลด\\
\mbox{และติดตั้ง} OakCoding ได้ตามขั้นตอนดังต่อไปนี้

\begin{enumerate}
\setlength{\itemsep}{0.7pt}
    \item เปิดแอป Terminal\marginnote{สำหรับความช่วยเหลือเกี่ยวกับแอป Terminal\\
    โปรดไปที่ \href{https://support.apple.com/th-th/guide/terminal/welcome/mac/}{support.apple.com/th-th/guide/terminal/welcome/mac/}}

    \oakparimagehere{sections/images/01/01-01_terminal-app.png}

    \item จากนั้นคุณจะต้องดาวน์โหลดไฟล์ทีี่จำเป็นด้วยคำสั่ง:

\begin{lstlisting}[numbers=none]
$ git clone \
> %\oakrepourl%
\end{lstlisting}

    หากคุณได้รับข้อความว่า ``\texttt{Username for 'https://github.com': }''\\
    ให้คุณทำการลงชื่อเข้าใช้ด้วยบัญชี GitHub ของคุณ\marginnote{บัญชี GitHub ของคุณจะต้องมีสิทธิ์เข้าถึง Git repository ของ OakCoding ด้วย
    ไม่เช่นนั้นคุณจะไม่สามารถดาวน์โหลดไฟล์ต่าง ๆ ได้}

    \item เมื่อการดาวน์โหลดเสร็จสิ้น ให้คุณตรวจสอบว่ามี directory ที่ชื่อว่า ``\mbox{\texttt{project211-oakcoding}}'' อยู่หรือไม่ด้วยคำสั่ง:

\begin{lstlisting}[numbers=none]
$ ls
\end{lstlisting}

    หากการดาวน์โหลดนั้นเสร็จสมบูรณ์ จะต้องมีชื่อ directory ปรากฏอยู่

\begin{lstlisting}[numbers=none]
$ ls
project211-oakcoding
\end{lstlisting}
\end{enumerate}

เมื่อคุณได้ดาวน์โหลดไฟล์ที่จำเป็นเสร็จสมบูรณ์แล้ว ให้คุณเปลี่ยน directory เข้าไปที่ ``\texttt{project211-oakcoding}'' ด้วยคำสั่ง:

\begin{lstlisting}[numbers=none]
$ cd project211-oakcoding
\end{lstlisting}

\clearpage

จากนั้น เมื่อคุณใช้คำสั่ง ``\texttt{ls run.sh}'' เพื่อตรวจสอบเนื้อหาภายใน directory คุณควรจะเห็นไฟล์หนึ่งที่มีชื่อว่า ``\texttt{run.sh}''

\begin{lstlisting}[numbers=none]
$ ls run.sh
run.sh
\end{lstlisting}

และนั่นก็คือไฟล์ที่จะช่วยให้คุณสามารถที่จะติดตั้งและเปิดใช้งานแอปได้ภายในคำสั่งเดียว

เราจะไปต่อที่ \oaksecref{subsec:launch-app} กัน

\clearpage

\subsection{ติดตั้งและเปิดใช้งานแอป}\label{subsec:launch-app}

\rule{0em}{1ex}

การติดตั้งแอปอาจมีการดาวน์โหลดสิ่งที่จำเป็นอีกเล็กน้อย
นั่นก็เพื่อให้แอปทำงานได้อย่างลื่นไหล
คุณจะต้องเชื่อมต่ออินเทอร์เน็ตอยู่ตลอดการติดตั้งแอป

คุณสามารถติดตั้งและเปิดใช้งานแอปได้ด้วยคำสั่ง:

\begin{lstlisting}[numbers=none]
$ sh run.sh
\end{lstlisting}

การติดตั้งอาจใช้เวลาครู่หนึ่ง ขึ้นอยู่กับหลายปัจจัย อย่างเช่น อินเทอร์เน็ตของคุณ
คอมพิวเตอร์ของคุณ หรือหากคุณไม่เคยเรียกใช้คำสั่ง ``\texttt{mvn}'' เพื่อสร้างหรือเปิดใช้งาน
project ใด ๆ มาก่อนบนคอมพิวเตอร์ของคุณ

หากการติดตั้งนั้นผ่านไปได้ด้วยดี แอป OakCoding จะเปิดขึ้นมาต้อนรับคุณ

\oakparimagehere{sections/images/01/01-03_app-welcome.png}

หน้าตาของแอปอาจแตกต่างจากที่คุณเห็นในคู่มือเล่มนี้เล็กน้อย
ขึ้นอยู่กับว่าคุณใช้งานระบบปฏิบัติการใด สำหรับ macOS ชุดปุ่มควบคุมหน้าต่างจะอยู่มุมบนซ้ายของแอป
ส่วนระบบปฏิบัติการอื่น ๆ ชุดปุ่มดังกล่าวจะอยู่ที่มุมบนขวา

OakCoding จะปรับตัวให้เข้ากับการใช้งานของคุณ บนระบบปฏิบัติการของคุณ

\clearpage

\subsection{การใช้งานทั่วไป}\label{subsec:general-usage}

\oakparimagehere{sections/images/01/01-03_app-welcome.png}

เลือก ``Get Started'' เพื่อเริ่มใช้งาน จากนั้นคุณจะพบกับหน้าลงชื่อเข้าใช้

\oakparimagehere{sections/images/01/01-04_app-login.png}

\begin{itemize}
    \item หากคุณมีบัญชีผู้ใช้ของแอปอยู่แล้ว
คุณสามารถลงชื่อเข้าใช้ด้วยการกรอกชื่อผู้ใช้ของคุณในช่อง ``Username'' และรหัสผ่านของคุณลงในช่อง ``Password''
แล้วเลือก ``Login'' ได้เลย ต่อไปให้คุณไปที่~\oaksecref{subsec:basic-usage}
    \item หากคุณไม่มีบัญชีผู้ใช้ของแอป และคุณเป็นนิสิตของมหาวิทยาลัยเกษตรศาสตร์ คุณสามารถสร้างบัญชีผู้ใช้\relax
ในสถานะผู้ใช้ทั่วไปได้ด้วยการเลือก ``Register'' แล้วไปที่ \oaksecref{subsubsec:register-consumer-acct}
\end{itemize}

\subsubsection{การสร้างบัญชีผู้ใช้ทั่วไป}\label{subsubsec:register-consumer-acct}

\oakparimagehere{sections/images/01/01-05_app-consumer-register-acct.png}

เมื่อคุณอยู่ในหน้าสร้างบัญชีผู้ใช้แล้ว เพื่อสร้างบัญชีของคุณ คุณจะต้อง:

\begin{enumerate}
    \item ใส่ข้อมูลส่วนตัวของคุณลงใน:
        \begin{description}
            \item[``Firstname''] ชื่อต้น/ชื่อจริง ของคุณ
            \item[``Lastname''] นามสกุลของคุณ
            \item[``Username''] \marginnote{คุณสามารถเปลี่ยนชื่อผู้ใช้ในภายหลังได้ แต่ชื่อผู้ใช้จะต้องไม่ซ้ำกับผู้ใช้อื่นในระบบในขณะนั้น}\parbox[t]{0.5\textwidth}{ตั้งชื่อผู้ใช้ของคุณ โดยชื่อผู้ใช้จะต้องประกอบไปด้วยตัวอักษรในภาษาอังกฤษพิมพ์เล็กและตัวเลขฮินดู-อาราบิกเท่านั้น ห้ามมีอักขระว่าง (whitespace character)}
            \item[``Password''] ตั้งรหัสผ่านบัญชีของคุณ
            \item[``Confirm Password''] ยืนยันรหัสผ่านบัญชีของคุณ
        \end{description}
    \item หากต้องการ คุณสามารถเลือกรูป profile สำหรับบัญชีของคุณได้ด้วยการเลือก ``Upload Profile Image''\marginnote{คุณสามารถตั้งรูป profile ในภายหลังได้}
    \item เพื่อยืนยันว่าคุณเป็นนิสิตของมหาวิทยาลัยเกษตรศาสตร์ ให้คุณทำเครื่องหมาย \checkmark{} ด้านหน้าข้อความ ``I am a student of Kasetsart University''
    \item เลือก ``Register'' เพื่อเสร็จสิ้นการสร้างบัญชี
\end{enumerate}

หากเกิดข้อผิดพลาดใด ๆ แอปจะแจ้งเตือนให้คุณทราบ
หากทุกอย่างผ่านไปด้วยดีและแอปแจ้งคุณว่า ``คุณได้ทำการสมัครสมาชิกเรียบร้อยแล้ว'' นั่นหมายความว่า คุณได้สร้างบัญชีผู้ใช้ของแอปเรียบร้อยแล้ว

เมื่อคุณเลือก ``OK'' แอปจะพาคุณไปที่หน้าลงชื่อเข้าใช้\\
คุณสามารถลงชื่อเข้าใช้ด้วยการกรอกชื่อผู้ใช้ของคุณในช่อง ``Username''\\
และรหัสผ่านของคุณลงในช่อง ``Password'' แล้วเลือก ``Login'' ได้เลย

\clearpage

\subsection{การใช้งานเบื้องต้น}\label{subsec:basic-usage}

คุณสามารถนำวิธีการใช้งานเบื้องต้นไปปรับใช้กับบัญชีผู้ใช้ได้ในทุกระดับสิทธิ์

\subsubsection{ตั้งค่าบัญชีผู้ใช้}\label{subsec:acct-settings}

\blindtext[3]

\subsubsection{เปลี่ยนรหัสผ่านบัญชีผู้ใช้}\label{subsec:changing-password}

\blindtext[3]

\clearpage

\subsection{การใช้งานสำหรับผู้ใช้ระบบในระดับสิทธิ์ต่าง ๆ}\label{subsec:role-specific-usage}
ความสามารถของ application ที่จำเพาะต่อผู้ใช้ระบบในแต่ละระดับสิทธิ์

\subsubsection{ผู้ดูแลระบบ (Admin)}\label{subsubsec:role-usage-admin}

\blindtext[3]

\subsubsection{เจ้าหน้าที่ (Staff)}\label{subsubsec:role-usage-staff}

\blindtext[3]

\subsubsection{นิสิต (Consumer)}\label{subsubsec:role-usage-consumer}

\oaksimsec{เข้าถึงและแจ้งเรื่องร้องเรียน}

\thebestimage{complaint list image}

คุณสามารถเข้าถึงเรื่องร้องเรียนทั้งหมดในระบบได้โดยการเลือก ``Complaints'' ที่แถบด้านข้าง

\pagebreak[3]

\oaksimsec{แจ้งเรื่องร้องเรียน}

คุณสามารถแจ้งเรื่องร้องเรียนได้โดยการ:

\begin{enumerate}
    \item เลือก ``Create Complaint''
    \item เลือก ``Catagory'' เพื่อระบุหมวดหมู่ของเรื่องร้องเรียนที่คุณกำลังจะแจ้ง
    \item ในช่อง ``Subject'' ระบุหัวเรื่องของเรื่องร้องเรียนสั้น ๆ
    \item ในช่อง ``Description'' ระบุรายละเอียดที่คุณต้องการจะแจ้ง
    \item หากต้องการ คุณสามารถแนบไฟล์หลักฐานได้โดยเลือก ?? อัพโหลด (ลูกศรชี้ขึ้นใต้คำว่า Evidence) ??
    \item เลือก ``Create'' เพื่อแจ้งเรื่องร้องเรียน
\end{enumerate}

\pagebreak[3]

\oaksimsec{ดูรายละเอียดของเรื่องร้องเรียน}

\thebestimage{detailed view of a complaint image}

เพื่อเข้าดูรายละเอียดของเรื่องร้องเรียนใด ๆ ให้คุณเลือกเรื่องร้องเรียนที่คุณต้องการจากตารางในหน้ารวมเรื่องร้องเรียน

\pagebreak[3]

\oaksimsec{ลงคะแนนเสียงให้กับเรื่องร้องเรียน}

\thebestimage{detailed view of a complaint image}

คุณสามารถลงคะแนนเสียงให้กับเรื่องร้องเรียนที่ผู้ใช้อื่น ๆ เป็นผู้แจ้งได้โดยเลือก ``Vote''
ที่มุมล่างขวาในหน้าแสดงรายละเอียดของเรื่องร้องเรียน
คุณสามารถลงคะแนนเสียงให้กับเรื่องร้องเรียน 1 เรื่องได้ไม่เกิน 1 ครั้ง

\clearpage
