\documentclass[../docs.tex]{subfiles}

\begin{document}
\section*{ข้อควรรู้}
\markboth{ข้อควรรู้}{}

ในส่วนนี้ เราจะอธิบายข้อมูลเบื้องต้นที่คุณควรรู้ก่อนเริ่มใช้งาน OakCoding

\rule{0em}{1ex}
\subsection*{OakCoding คืออะไร}
\textbf{OakCoding} คือ desktop application --- เรียกสั้น ๆ ได้ว่า \textit{app} หรือ \textit{แอป} --- ที่ใช้สำหรับ\\
แจ้งเรื่องร้องเรียนของนิสิตมหาวิทยาลัยเกษตรศาสตร์

\rule{0em}{1ex}
\subsection*{ฉันจะใช้งานแอป OakCoding ได้อย่างไร}
หากคุณต้องการใช้งานแอป คุณจะต้องมีบัญชีผู้ใช้ของแอป โดยที่บัญชีผู้ใช้จะมี 3 ลักษณะใหญ่ ๆ ขึ้นอยู่กับว่าคุณมีสถานะเป็นอะไรในมหาวิทยาลัยเกษตรศาสตร์
และคุณมีสถานะเป็นอะไรภายในแอป

\rule{0em}{1ex}
\subsection*{ทำความเข้าใจบัญชี สถานะ และระดับสิทธิ์ภายในแอป}
ภายในแอป จะมีการแบ่งสถานะของผู้ใช้ --- หรือเรียกอีกอย่างหนึ่งว่าระดับสิทธิ์ของผู้ใช้ --- ออกเป็น 3 ระดับ
ผู้ใช้ในแต่ละระดับจะมีสิทธิ์เข้าถึงความสามารถ (feature) ของแอปที่แตกต่างกัน นั่นก็เพื่อความเป็นระเบียบในการบริหารจัดการระบบภายในแอป

\newpage

\subsubsection*{บัญชีและระดับสิทธิ์ของผู้ใช้}
ระดับสิทธิ์ของผู้ใช้ประกอบด้วย 3 ระดับ \marginpar{ในปัจจุบัน บัญชีผู้ใช้ที่ถูกสร้างขึ้นในระบบจะไม่สามารถลบออกได้} ได้แก่ ผู้ดูแลระบบ (Admin), เจ้าหน้าที่ (Staff), และ นิสิต (Consumer)

\begin{enumerate}
    \item ผู้ดูแลระบบ (Admin)

        ผู้ใช้ที่มีระดับสิทธิ์สูงสุดในระบบ มีสิทธิ์ในการจัดการ (manipulate) เนื้อหา (content)
        และบัญชีผู้ใช้ในระดับสิทธิ์อื่น ๆ เช่น พิจารณาลบเรื่องร้องเรียน (complaint) ที่ถูกรายงานว่ามีเนื้อหาไม่เหมาะสม, พิจารณาระงับ (suspend) บัญชีผู้ใช้
        ของนิสิตที่ถูกรายงานว่ามีพฤติกรรมไม่เหมาะสมในการใช้งานระบบ, และสร้างบัญชีผู้ใช้สำหรับเจ้าหน้าที่ เป็นต้น\\
        \rule{0em}{1ex}\\
        ในระบบจะมีบัญชีของผู้ดูแลระบบได้เพียง 1 บัญชี โดยเราได้จัดเตรียมไว้ให้แล้ว คุณจึงไม่จำเป็นต้องสร้างบัญชี
        ของผู้ดูแลระบบเอง
    \item เจ้าหน้าที่ (Staff)

        ผู้ใช้ที่มีหน้าที่ในการจัดการเรื่องร้องเรียน ในส่วนที่ตนเองรับผิดชอบ บัญชีผู้ใช้ของเจ้าหน้าที่จะถูกสร้าง
        (register) โดยผู้ดูแลระบบเท่านั้น ไม่สามารถสร้างบัญชีด้วยตนเองได้
    \item นิสิต (Consumer)

        ผู้ใช้ทั่วไปซึ่งเป็นนิสิตของมหาวิทยาลัยเกษตรศาสตร์ สามารถแจ้ง เข้าถึง หรือติดตามเรื่องร้องเรียน
        ในระบบได้ทั้งนี้ หากผู้ใช้ระบบในระดับสิทธิ์นี้มีพฤติกรรมการใช้งานที่ไม่เหมาะสม หรือนำเข้าเนื้อหาที่ไม่เหมาะสม
        อาจถูกรายงาน (report) เพื่อให้ผู้ดูแลระบบพิจารณาระงับบัญชีผู้ใช้ หรือลบเนื้อหาที่ไม่เหมาะสมได้ ตามลำดับ
        ในกรณีการถูกระงับบัญชีผู้ใช้ของนิสิต นิสิตที่ถูกระงับบัญชีจะไม่มีสิทธิ์ในการเข้าสู่ระบบ (login)
        แต่สามารถส่งคำร้อง (request) เพื่อให้ผู้ดูแลระบบพิจารณาปลดระงับบัญชีได้
\end{enumerate}

\clearpage

\section*{ข้อมูลทางเทคนิค}
\markboth{ข้อมูลทางเทคนิค}{}

OakCoding ถูกพัฒนาด้วยภาษา Java โดยใช้ Java Development Kit (JDK) version 17 ร่วมกับ JavaFX version 17
และใช้ \href{https://maven.apache.org/}{Maven} ในการ automate building process

\rule{0em}{1ex}
\subsection*{โครงสร้าง Directory ของ Project}

โครงสร้าง directory ของ project ถูกดัดแปลงจาก\href{https://maven.apache.org/guides/introduction/introduction-to-the-standard-directory-layout.html}{โครงสร้าง repository มาตรฐานของ Maven}

\setlength{\parindent}{0em}
\setlength{\columnsep}{2pt}
\begin{lstlisting}[name={โครงสร้าง Directory ของ Project},numbers=none]
<PROJECT ROOT>/
    data/
    docs/
    src/
    submit/
    mvnw*
    mvnw.cmd*
    oakcommit.sh*
    pom.xml
    README.md
    run.sh*
\end{lstlisting}

\subsection*{คำอธิบาย Directory ของ Project}

\begin{tabularx}{\textwidth}{ @{} p{6em} @{} p{\textwidth-8em} @{} }
\textbf{ไฟล์ / Directory}    & \textbf{คำอธิบาย} \\
\texttt{data/}              & ข้อมูลของผู้ใช้ระบบ ได้แก่ ไฟล์ CSV และไฟล์ที่ผู้ใช้ทำการนำเข้า (import) สู่ระบบ \\
\texttt{docs/}              & Source code การร่างคู่มือการใช้งาน app (เอกสารที่คุณกำลังอ่านอยู่นี้) \\
\texttt{src/}               & Application source code และไฟล์ resource ที่ใช้ใน app \\
\texttt{submit/}            & ไฟล์ executable และไฟล์ที่เกี่ยวข้องกับการเปิดใช้งาน app \\
\texttt{mvnw*}              & \href{https://maven.apache.org/wrapper/}{Maven wrapper script} สําหรับระบบปฏิบัติการ UNIX และ UNIX-like \\
\texttt{mvnw.cmd*}          & \href{https://maven.apache.org/wrapper/}{Maven wrapper script} สําหรับระบบปฏิบัติการ Microsoft Windows \\
\texttt{oakcommit.sh*}      & Script สำหรับป้องกัน Git commit confliction ในระหว่างการพัฒนา app \\
\texttt{pom.xml}            & \href{https://maven.apache.org/guides/introduction/introduction-to-the-pom.html}{Project Object Model ของ Maven} \\
\texttt{README.md}          & ไฟล์แสดงข้อมูลเกี่ยวกับ project รวมถึงคำอธิบายเกี่ยวกับไฟล์ และ directory ต่าง ๆ \\
\texttt{run.sh*}            & Script สำหรับเปิดใช้งาน app
\end{tabularx}
\end{document}
