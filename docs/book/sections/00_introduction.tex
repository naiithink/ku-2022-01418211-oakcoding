\section*{บทนำ}

\subsection*{เกี่ยวกับ Application}

OakCoding เป็น desktop application ที่ใช้สำหรับแจ้งเรื่องร้องเรียนของนิสิตมหาวิทยาลัยเกษตรศาสตร์\\
ภายใน application --- หรือเรียกสั้น ๆ ว่า \textit{app} --- จะมีการแบ่งระดับสิทธิ์ของผู้ใช้งานออกเป็น 3 ระดับ ผู้ใช้ในแต่ละระดับ\\
จะมีสิทธิ์เข้าถึงความสามารถ (feature) ของ app ที่แตกต่างกันเพื่อความเป็นระเบียบในการบริหารจัดการระบบภายใน app

\subsubsection*{บัญชีและระดับสิทธิ์ของผู้ใช้ระบบ}
ระดับสิทธิ์ของผู้ใช้ระบบประกอบด้วย 3 ระดับสิทธิ์ ได้แก่ ผู้ดูแลระบบ (Admin), เจ้าหน้าที่ (Staff), และ นิสิต (Consumer)

\begin{enumerate}
    \item ผู้ดูแลระบบ (Admin)

        ผู้ใช้ที่มีระดับสิทธิ์สูงสุดในระบบ มีสิทธิ์ในการจัดการ (manipulate) เนื้อหา (content) และบัญชีผู้ใช้ในระดับสิทธิ์อื่น ๆ\\
        เช่น พิจารณาลบเรื่องร้องเรียน (complaint) ที่ถูกรายงานว่ามีเนื้อหาไม่เหมาะสม, พิจารณาระงับ (suspend) บัญชีผู้ใช้\\
        ของนิสิตที่ถูกรายงานว่ามีพฤติกรรมไม่เหมาะสมในการใช้งานระบบ, และสร้างบัญชีผู้ใช้สำหรับเจ้าหน้าที่ เป็นต้น\\
        ในระบบจะมีบัญชีของผู้ดูแลระบบได้เพียง 1 บัญชี โดยจะถูกจัดเตรียมไว้ให้แล้วโดยทีมผู้พัฒนา application
    \item เจ้าหน้าที่ (Staff)

        ผู้ใช้ที่มีหน้าที่ในการจัดการเรื่องร้องเรียน ในส่วนที่ตนเองรับผิดชอบ บัญชีผู้ใช้ของเจ้าหน้าที่จะถูกสร้าง\\
        (register) โดยผู้ดูแลระบบเท่านั้น ไม่สามารถสร้างบัญชีด้วยตนเองได้
    \item นิสิต (Consumer)

        ผู้ใช้ทั่วไปซึ่งเป็นนิสิตของมหาวิทยาลัยเกษตรศาสตร์ สามารถแจ้ง เข้าถึง หรือติดตามเรื่องร้องเรียนในระบบได้\\
        ทั้งนี้ หากผู้ใช้ระบบในระดับสิทธิ์นี้มีพฤติกรรมการใช้งานที่ไม่เหมาะสม หรือนำเข้าเนื้อหาที่ไม่เหมาะสม\\
        อาจถูกรายงาน (report) เพื่อให้ผู้ดูแลระบบพิจารณาระงับบัญชีผู้ใช้ หรือลบเนื้อหาที่ไม่เหมาะสมได้ ตามลำดับ\\
        ในกรณีการถูกระงับบัญชีผู้ใช้ของนิสิต นิสิตที่ถูกระงับบัญชีจะไม่มีสิทธิ์ในการเข้าสู่ระบบ (login)\\
        แต่สามารถส่งคำร้อง (request) เพื่อให้ผู้ดูแลระบบพิจารณาปลดระงับบัญชีได้
\end{enumerate}

\noindent\framebox[\textwidth][c]{ในปัจจุบัน บัญชีผู้ใช้ที่ถูกสร้างขึ้นในระบบจะไม่สามารถลบออกได้}


\pagebreak
\subsection*{เกี่ยวกับการพัฒนา Application}

OakCoding ถูกพัฒนาด้วยภาษา Java โดยใช้ Java Development Kit (JDK) version 17 ร่วมกับ JavaFX version 17\\
และใช้ \href{https://maven.apache.org/}{Maven} ในการ automate building process

\subsubsection*{โครงสร้าง Directory ของ Project}

โครงสร้าง directory ของ project ถูกดัดแปลงจาก\href{https://maven.apache.org/guides/introduction/introduction-to-the-standard-directory-layout.html}{โครงสร้าง repository มาตรฐานของ Maven}

\setlength{\parindent}{0em}
\setlength{\columnsep}{2pt}
\begin{lstlisting}[name={Project Directory Layout},numbers=none]
<PROJECT ROOT>/
    data/
    docs/
    src/
    submit/
    mvnw*
    mvnw.cmd*
    oakcommit.sh*
    pom.xml
    README.md
    run.sh*
\end{lstlisting}

\subsubsection*{คำอธิบาย Directory ของ Project}

\begin{tabularx}{\textwidth}{ p{8em} p{\textwidth-8em} }
\textbf{ชื่อไฟล์ / Directory}      & \textbf{คำอธิบาย} \\
% \hline
\texttt{data/}              & ข้อมูลของผู้ใช้ระบบ ได้แก่ ไฟล์ CSV และไฟล์ที่ผู้ใช้ทำการนำเข้า (import) สู่ระบบ \\
\texttt{docs/}              & Source code การร่างคู่มือการใช้งาน application (เอกสารที่ท่านกำลังอ่านอยู่นี้) \\
\texttt{src/}               & Application source code และไฟล์ resource ที่ใช้ใน application \\
\texttt{submit/}            & ไฟล์ executable และไฟล์ที่เกี่ยวข้องกับการเปิดใช้งาน application \\
\texttt{mvnw*}              & \href{https://maven.apache.org/wrapper/}{Maven wrapper script} สําหรับระบบปฏิบัติการ UNIX และ UNIX-like \\
\texttt{mvnw.cmd*}          & \href{https://maven.apache.org/wrapper/}{Maven wrapper script} สําหรับระบบปฏิบัติการ Microsoft Windows \\
\texttt{oakcommit.sh*}      & Script สำหรับป้องกัน Git commit confliction ในระหว่างการพัฒนา application \\
\texttt{pom.xml}            & \href{https://maven.apache.org/guides/introduction/introduction-to-the-pom.html}{Project Object Model ของ Maven} \\
\texttt{README.md}          & ไฟล์แสดงข้อมูลเกี่ยวกับ project รวมถึงคำอธิบายเกี่ยวกับไฟล์ และ directory ต่าง ๆ \\
\texttt{run.sh*}            & Script สำหรับเปิดใช้งาน application
\end{tabularx}
