\section*{ข้อควรรู้}
\markboth{ข้อควรรู้}{}

ในส่วนนี้ เราจะอธิบายข้อมูลเบื้องต้นที่คุณควรรู้ก่อนเริ่มใช้งาน OakCoding

\rule{0em}{1ex}
\subsection*{OakCoding คืออะไร}
\textbf{OakCoding} คือ desktop application --- หรือเรียกสั้น ๆ ได้ว่า \emph{แอป} --- ที่ใช้สำหรับ
แจ้งเรื่องร้องเรียนของนิสิตมหาวิทยาลัยเกษตรศาสตร์

\rule{0em}{1ex}
\subsection*{ฉันจะใช้งานแอป OakCoding ได้อย่างไร}
หากคุณต้องการใช้งานแอป คุณจะต้องมีบัญชีผู้ใช้ของแอป โดยที่บัญชีผู้ใช้จะแบ่งออกเป็น 3 ลักษณะใหญ่ ๆ ขึ้นอยู่กับว่าคุณมีสถานะเป็นอะไรในมหาวิทยาลัยเกษตรศาสตร์
และคุณมีสถานะเป็นอะไรภายในแอป

\rule{0em}{1ex}
\subsection*{ทำความเข้าใจบัญชี สถานะ และระดับสิทธิ์ภายในแอป}
ภายในแอป จะมีการแบ่งสถานะของผู้ใช้ --- หรือเรียกอีกอย่างหนึ่งว่าระดับสิทธิ์ของผู้ใช้ --- ออกเป็น 3 ระดับ
ผู้ใช้ในแต่ละระดับจะมีสิทธิ์เข้าถึงความสามารถของแอปที่แตกต่างกัน นั่นก็เพื่อความเป็นระเบียบในการบริหารจัดการระบบภายในแอป

\newpage

\subsubsection*{บัญชีและระดับสิทธิ์ของผู้ใช้}
ระดับสิทธิ์ของผู้ใช้ประกอบด้วย 3 ระดับ \marginpar{ในปัจจุบัน บัญชีผู้ใช้ที่ถูกสร้างขึ้นในระบบจะไม่สามารถลบออกได้} ได้แก่ ผู้ดูแลระบบ, เจ้าหน้าที่, และนิสิต


\begin{description}
    \item[ผู้ดูแลระบบ (Admin)]
        \begin{minipage}[t]{\textwidth - \labelwidth - \labelsep}
        ผู้ใช้ที่มีระดับสิทธิ์สูงสุดในระบบ มีสิทธิ์ในการจัดการ (manipulate)
        เนื้อหา (content) และบัญชีผู้ใช้ในระดับสิทธิ์อื่น ๆ เช่น
            \begin{itemize}
                \item พิจารณาลบเรื่องร้องเรียน (complaint) ที่ถูกรายงานว่า
                    มีเนื้อหาที่\linebreak[3]ไม่เหมาะสม
                \item พิจารณาระงับ (suspend) บัญชีผู้ใช้ของนิสิตที่ถูกรายงานว่า
                    มีพฤติกรรมที่\linebreak[3]ไม่เหมาะสมในการใช้งานระบบ
                \item และสร้างบัญชีผู้ใช้สำหรับเจ้าหน้าที่ เป็นต้น
            \end{itemize}

        ในระบบจะมีบัญชีของผู้ดูแลระบบได้เพียง 1 บัญชี โดยเราได้จัดเตรียมไว้ให้คุณแล้ว คุณจึงไม่จำเป็นต้องสร้างบัญชีของผู้ดูแลระบบเอง
        \end{minipage}
    \item[เจ้าหน้าที่ (Staff)]
        \begin{minipage}[t]{\textwidth - \labelwidth - \labelsep}
        ผู้ใช้ที่มีหน้าที่ในการจัดการเรื่องร้องเรียนในส่วนที่ตนเองรับผิดชอบ บัญชีผู้ใช้ของเจ้าหน้าที่จะถูกสร้าง
        (register) โดยผู้ดูแลระบบเท่านั้น \mbox{ไม่สามารถสร้างบัญชี}ด้วยตนเองได้
        \end{minipage}
    \item[นิสิต (Consumer)]
        \begin{minipage}[t]{\textwidth - \labelwidth - \labelsep}
        ผู้ใช้ทั่วไปซึ่งเป็นนิสิตของมหาวิทยาลัยเกษตรศาสตร์ สามารถแจ้ง เข้าถึง\linebreak[3]
        หรือติดตามเรื่องร้องเรียนในระบบได้ ทั้งนี้ หากผู้ใช้ระบบในระดับสิทธิ์นี้มีพฤติกรรมการใช้งานที่ไม่เหมาะสม หรือนำเข้าเนื้อหาที่ไม่เหมาะสม
        \mbox{อาจถูกรายงาน (report)} เพื่อให้ผู้ดูแลระบบพิจารณาระงับบัญชีผู้ใช้ หรือลบเนื้อหาที่ไม่เหมาะสมได้ ตามลำดับ

        \vspace{1.5ex}

        ในกรณีการถูกระงับบัญชีผู้ใช้ของนิสิต นิสิตที่ถูกระงับบัญชีจะไม่มีสิทธิ์ในการเข้าสู่ระบบ (login)
        แต่สามารถส่งคำร้อง (request) เพื่อให้ผู้ดูแลระบบพิจารณาปลดระงับบัญชีได้
        \end{minipage}
\end{description}

\clearpage

\section*{ข้อมูลทางเทคนิค}
\markboth{ข้อมูลทางเทคนิค}{}

OakCoding ถูกพัฒนาด้วยภาษา Java โดยใช้ Java Development Kit (JDK) 17
\mbox{ร่วมกับ} JavaFX 17 และใช้ \href{https://maven.apache.org/}{Maven} ในการ automate building process

\rule{0em}{1ex}
\subsection*{โครงสร้าง Directory ของ Project}

โครงสร้าง directory ของ project ถูกดัดแปลงจาก\href{https://maven.apache.org/guides/introduction/introduction-to-the-standard-directory-layout.html}{โครงสร้าง repository มาตรฐานของ Maven}

\setlength{\parindent}{0em}
\setlength{\columnsep}{2pt}
\begin{lstlisting}[title={โครงสร้าง Directory ของ Project},numbers=none]
<PROJECT ROOT>/
    data/
    docs/
    src/
    submit/
    mvnw*
    mvnw.cmd*
    oakcommit.sh*
    pom.xml
    README.md
    run.sh*
\end{lstlisting}

\clearpage

\subsection*{คำอธิบาย Directory ของ Project}

\begin{description}[labelwidth = 0.8\labelwidth]
    \item[\texttt{data/}]
        \begin{minipage}[t]{\textwidth - \labelwidth - \labelsep}
        ข้อมูลของผู้ใช้ระบบ ได้แก่ ไฟล์ CSV และไฟล์ที่ผู้ใช้ทำการนำเข้าสู่ระบบ
        \end{minipage}
    \item[\texttt{docs/}]
        \begin{minipage}[t]{\textwidth - \labelwidth - \labelsep}
        Source code การร่างคู่มือการใช้งานแอป (เอกสารที่คุณกำลังอ่านอยู่นี้)
        \end{minipage}
    \item[\texttt{src/}]
        \begin{minipage}[t]{\textwidth - \labelwidth - \labelsep}
        Application source code และไฟล์ resource ที่ใช้ในแอป
        \end{minipage}
    \item[\texttt{submit/}]
        \begin{minipage}[t]{\textwidth - \labelwidth - \labelsep}
        ไฟล์ executable และไฟล์ที่เกี่ยวข้องกับการเปิดใช้งานแอป
        \end{minipage}
    \item[\texttt{mvnw*}]
        \begin{minipage}[t]{\textwidth - \labelwidth - \labelsep}
        \href{https://maven.apache.org/wrapper/}{Maven wrapper script} สําหรับระบบปฏิบัติการ UNIX
        \end{minipage}
    \item[\texttt{mvnw.cmd*}]
        \begin{minipage}[t]{\textwidth - \labelwidth - \labelsep}
        \href{https://maven.apache.org/wrapper/}{Maven wrapper script} สําหรับระบบปฏิบัติการ Microsoft Window
        \end{minipage}
    \item[\texttt{oakcommit.sh*}]
        \begin{minipage}[t]{\textwidth - \labelwidth - \labelsep}
        Script สำหรับป้องกัน Git commit confliction ในระหว่างการพัฒนาแอป
        \end{minipage}
    \item[\texttt{pom.xml}]
        \begin{minipage}[t]{\textwidth - \labelwidth - \labelsep}
        \href{https://maven.apache.org/guides/introduction/introduction-to-the-pom.html}{Project Object Model ของ Maven}
        \end{minipage}
    \item[\texttt{README.md}]
        \begin{minipage}[t]{\textwidth - \labelwidth - \labelsep}
        ไฟล์แสดงข้อมูลเกี่ยวกับ project รวมถึงคำอธิบายเกี่ยวกับไฟล์และ directory ต่าง ๆ
        \end{minipage}
    \item[\texttt{run.sh*}]
        \begin{minipage}[t]{\textwidth - \labelwidth - \labelsep}
        Script สำหรับเปิดใช้งานแอป
        \end{minipage}
\end{description}
