\section{รายละเอียดไฟล์ CSV}

\subsection{ไฟล์ CSV ใน \href{https://github.com/CS211-651/project211-oakcoding/tree/9397d355461933fb007261e2ee97445ea93eacc1/data}{\texttt{data\slash}} Directory}

\begin{lstlisting}[name={ไฟล์ CSV ใน data/ Directory},numbers=none]
data/
    ...
    deps/
        <DEPARTMENTS' DIRECTORIES>/
            info.csv
            members.csv
            tags.csv
        ...
        deps.csv
    issues/
        complaints/
            evidences/
                evidences.csv
            categories.csv
            complaints.csv
        reports/
            reports.csv
    users/
        _ROOT/
            info.csv
        <USERS' DIRECTORIES>/
            info.csv
        ...
        requests.csv
    passwd.csv
    sessions.csv
    users.csv
\end{lstlisting}

\clearpage

ไฟล์ CSV ใน \texttt{data/} directory จะมีรูปแบบร่วมกันคือ
ในแต่ละไฟล์จะมีหัวตาราง (header) และมีส่วนที่เป็นเนื้อหาของไฟล์ (body)

% \oak
% "UID","USER_NAME","ROLE","IS_ACTIVE","LOGIN_ATTEMPT"
% "1666340062670-staffuser","qq","STAFF","true","0"

% \begin{tabular}{@{} }
% \end{tabular}
\subsection{รายละเอียดของไฟล์ CSV แต่ละไฟล์}
\noindent\blindtext[3]
