\section{รายละเอียดไฟล์ CSV}

\subsection{การตีความไฟล์ CSV ใน \href{https://github.com/CS211-651/project211-oakcoding/tree/9397d355461933fb007261e2ee97445ea93eacc1/data}{\texttt{data\slash}~Directory}}

ไฟล์ CSV ใน \texttt{data\slash}~directory จะมีรูปแบบร่วมกันคือ
ในแต่ละไฟล์จะมีหัวตาราง (header) และมีส่วนที่เป็นเนื้อหาของไฟล์ (body)

ยกตัวอย่างเช่น ข้อมูลในตาราง People ต่อไปนี้:

\begin{table}[!htb]
\vspace{-\baselineskip}
\begin{tabular}{| l | l | l | l | l |}
\hline
\textbf{UID}    & \textbf{NAME}     & \textbf{EMAIL}\\
\hline\hline
0001            & Bob               & bob@example.com\\
\hline
0002            & Alice             & alice@example.com\\
\hline
\end{tabular}
\vspace{-\baselineskip}\caption{People}\label{tab:people}%
\end{table}

สามารถเก็บเป็นไฟล์ CSV ได้เป็น

\begin{lstlisting}[caption={\texttt{people.csv}},label={lst:people.csv}]
"UID","NAME","EMAIL"
"0001","Bob","bob@example.com"
"0002","Alice","alice@example.com"
\end{lstlisting}

header\marginnote{header ของตารางของไฟล์ CSV ใน project จะอยู่ที่บรรทัดแรกของไฟล์เสมอและจะถูกเขียนด้วยตัวอักษรภาษาอังกฤษพิมพ์ใหญ่ (uppercase) ทั้งหมดโดยที่ไม่มีอักขระว่าง (non-whitespace characters)} ของตารางมีเอาไว้เพื่อกำหนดว่าข้อมูลใน column/field นั้น ๆ คืออะไร
นั่นก็เพื่ออำนวยให้ทั้งคนและแอปสามารถตีความข้อมูลได้

ส่วน body ของตารางมีเอาไว้เพื่อเก็บข้อมูล โดยที่ข้อมูล 1 entity จะกินพื้นที่ 1 row/record ในตาราง
และจะวางตัวเป็น tuple; รูปแบบการ represent ข้อมูลอย่างมีลำดับ อย่างเช่น $ (x, y) $ จะเป็นคนละ entity กับ $ (y, x) $

และเมื่อต้องตีความข้อมูลจากไฟล์ \nameref{lst:people.csv} ก็สามารถตีความได้ดังนี้

\begin{enumerate}
    \item ไฟล์ที่มีชื่อว่า \texttt{people.csv} เป็นไฟล์ CSV ที่มี 3 columns
    \item แต่ละ column ในไฟล์แสดงข้อมูล \texttt{UID}, \texttt{NAME}, และ \texttt{EMAIL} ตามลำดับ
    \item ในไฟล์นั้นมี 1 record ที่ represent ข้อมูล:\marginnote{เริ่มตั้งแต่บรรทัดที่ 2 ของไฟล์ ลำดับของบรรทัดนั้นไม่มีนัยสำคัญ}
        \begin{itemize}
            \item \texttt{UID}: 0001
            \item \texttt{NAME}: Bob
            \item \texttt{EMAIL}: bob@example.com
        \end{itemize}
    \item และในไฟล์เดียวกันก็มีอีก 1 record ที่ represent ข้อมูล:
        \begin{itemize}
            \item \texttt{UID}: 0002
            \item \texttt{NAME}: Alice
            \item \texttt{EMAIL}: alice@example.com
        \end{itemize}
\end{enumerate}

คุณสามารถนำวิธีการตีความไฟล์ CSV นี้ไปปรับใช้กับไฟล์ CSV ใด ๆ ก็ได้ใน \texttt{data/} directory ของ project

\clearpage

\subsection{ไฟล์ CSV ใน \href{https://github.com/CS211-651/project211-oakcoding/tree/9397d355461933fb007261e2ee97445ea93eacc1/data}{\texttt{data\slash}~Directory}}

นี่คือโครงสร้างคร่าว ๆ เพื่อแสดงที่อยู่ของไฟล์ CSV ใน data/ directory

\begin{lstlisting}[caption={ไฟล์ CSV ใน data/ Directory},numbers=none]
data/
    ...
    deps/
        *<DEPARTMENT>/
            info.csv -------------------------------% \hyperref[subsubsec:csv-f1]{$ F_{1} $} %
            members.csv ----------------------------% \hyperref[subsubsec:csv-f2]{$ F_{2} $} %
            tags.csv -------------------------------% \hyperref[subsubsec:csv-f3]{$ F_{3} $} %
        ...
        deps.csv -----------------------------------% \hyperref[subsubsec:csv-f4]{$ F_{4} $} %
    issues/
        complaints/
            evidences/
                evidences.csv ----------------------% \hyperref[subsubsec:csv-f5]{$ F_{5} $} %
            categories.csv -------------------------% \hyperref[subsubsec:csv-f6]{$ F_{6} $} %
            complaints.csv -------------------------% \hyperref[subsubsec:csv-f7]{$ F_{7} $} %
        reports/
            reports.csv ----------------------------% \hyperref[subsubsec:csv-f8]{$ F_{8} $} %
    users/
        *<USER>/
            info.csv -------------------------------% \hyperref[subsubsec:csv-f9]{$ F_{9} $} %
        ...
        requests.csv -------------------------------% \hyperref[subsubsec:csv-f10]{$ F_{10} $} %
    passwd.csv -------------------------------------% \hyperref[subsubsec:csv-f11]{$ F_{11} $} %
    sessions.csv -----------------------------------% \hyperref[subsubsec:csv-f12]{$ F_{12} $} %
    users.csv --------------------------------------% \hyperref[subsubsec:csv-f13]{$ F_{13} $} %
\end{lstlisting}

% UNDERSCORED VERSION
% \begin{lstlisting}[name={ไฟล์ CSV ใน data/ Directory},numbers=none]
% data/
%     ...
%     deps/
%         *<DEPARTMENT>/
%             info.csv _______________________________% \hyperref[subsubsec:csv-f1]{$ F_{1} $} %
%             members.csv ____________________________% \hyperref[subsubsec:csv-f2]{$ F_{2} $} %
%             tags.csv _______________________________% \hyperref[subsubsec:csv-f3]{$ F_{3} $} %
%         ...
%         deps.csv ___________________________________% \hyperref[subsubsec:csv-f4]{$ F_{4} $} %
%     issues/
%         complaints/
%             evidences/
%                 evidences.csv ______________________% \hyperref[subsubsec:csv-f5]{$ F_{5} $} %
%             categories.csv _________________________% \hyperref[subsubsec:csv-f6]{$ F_{6} $} %
%             complaints.csv _________________________% \hyperref[subsubsec:csv-f7]{$ F_{7} $} %
%         reports/
%             reports.csv ____________________________% \hyperref[subsubsec:csv-f8]{$ F_{8} $} %
%     users/
%         *<USER>/
%             info.csv _______________________________% \hyperref[subsubsec:csv-f9]{$ F_{9} $} %
%         ...
%         requests.csv _______________________________% \hyperref[subsubsec:csv-f10]{$ F_{10} $} %
%     passwd.csv _____________________________________% \hyperref[subsubsec:csv-f11]{$ F_{11} $} %
%     sessions.csv ___________________________________% \hyperref[subsubsec:csv-f12]{$ F_{12} $} %
%     users.csv ______________________________________% \hyperref[subsubsec:csv-f13]{$ F_{13} $} %
% \end{lstlisting}

และเพื่อความสะดวกในการอ้างอิง เราจะให้ $ F_n $ แทนไฟล์ \texttt{*.csv} ที่ $ n $

\clearpage

\subsection{รายละเอียดของไฟล์ CSV แต่ละไฟล์}


% F1
\begin{minipage}{\textwidth}
\subsubsection[\texttt{data/deps/DEPARTMENT/info.csv}]{\texorpdfstring{$ F_{1} $}{File \#1}: \texorpdfstring{\parbox[t]{0.5\textwidth}{\texttt{data/deps/\\DEPARTMENT/info.csv}}}{deps/DEPARTMENT/info.csv}}\label{subsubsec:csv-f1}
\phantomsection

ข้อมูลทั่วไปของหน่วยงานผู้ใช้ Staff

\begin{tabular}[!hbt]{| r | >{\ttfamily}p{15ex}<{\rmfamily} | p{36ex} |}
\hline
Column \#       & \rmfamily Column Title                & คำอธิบาย\\
\hline
1               & DEP\textunderscore{}ID                & หมายเลขประจำหน่วยงาน\\
2               & DEF\textunderscore{}NAME              & ชื่อของหน่วยงาน\\
\hline
\end{tabular}

\oaksimsec{ตัวอย่างข้อมูล}

\begin{lstlisting}[caption={\texttt{data/deps/human\textunderscore{}resources/info.csv}}]
"DEP_ID","DEP_NAME"
"9999999999999-department","human_resources"
\end{lstlisting}

\begin{description}[labelwidth=*]
    \item[$ \Rightarrow $] หน่วยงานหมายเลข ``\texttt{9999999999999-department}'' เป็นหน่วยงานที่มีชื่อว่า ``human\textunderscore{}resources''
\end{description}
\end{minipage}


\vspace{3\baselineskip}


% F2
\begin{minipage}{\textwidth}
\subsubsection[\texttt{data/deps/DEPARTMENT/members.csv}]{\texorpdfstring{$ F_{2} $}{File \#2}: \texorpdfstring{\parbox[t]{0.5\textwidth}{\texttt{data/deps/\\DEPARTMENT/members.csv}}}{deps/DEPARTMENT/members.csv}}\label{subsubsec:csv-f2}
\phantomsection

ตารางของผู้ใช้ Staff ซึ่งเป็นสมาชิกในหน่วยงานผู้ใช้ Staff

\begin{tabular}[!hbt]{| r | >{\ttfamily}p{15ex}<{\rmfamily} | p{36ex} |}
\hline
Column \#       & \rmfamily Column Title                & คำอธิบาย\\
\hline
1               & UID                                   & หมายเลขประจำตัวผู้ใช้ของ Staff ที่เป็นสมาชิกของหน่วยงาน \\
2               & DATE\textunderscore{}ADDED            & เวลาที่เพิ่มสมาชิกคนนั้น ๆ เข้าสู่หน่วยงาน\\
\hline
\end{tabular}

\oaksimsec{ตัวอย่างข้อมูล}

\begin{lstlisting}[caption={\texttt{data/deps/human\textunderscore{}resources/members.csv}}]
"UID","DATE_ADDED"
"1000000000000-staffuser","0"
\end{lstlisting}

\begin{description}[labelwidth=*]
    \item[$ \Rightarrow $] ในหน่วยงานที่มีชื่อว่า ``human\textunderscore{}resources'' มีผู้ใช้สมาชิกอยู่ 1 ผู้ใช้ที่มี
\mbox{หมายเลขประจำตัวผู้ใช้} ``\texttt{1000000000000-staffuser}'' โดยถูกเพิ่มเข้ามา\relax
เมื่อเวลา 00:00:00 UTC ของวันพฤหัสบดีที่ 1 มกราคม ค.ศ. 1970
\end{description}
\end{minipage}


\vspace{3\baselineskip}


% F3
\begin{minipage}{\textwidth}
\subsubsection[\texttt{data/deps/DEPARTMENT/tags.csv}]{\texorpdfstring{$ F_{3} $}{File \#3}: \texorpdfstring{\parbox[t]{0.5\textwidth}{\texttt{data/deps/\\DEPARTMENT/tags.csv}}}{data/deps/DEPARTMENT/tags.csv}}\label{subsubsec:csv-f3}
\phantomsection

ตารางของหมวดหมู่เรื่องร้องเรียนในความรับผิดชอบทั้งหมดของหน่วยงานผู้ใช้ Staff

\begin{tabular}[!hbt]{| r | >{\ttfamily}p{15ex}<{\rmfamily} | p{36ex} |}
\hline
Column \#       & \rmfamily Column Title                & คำอธิบาย\\
\hline
1               & TAG                                   & หมวดหมู่เรื่องร้องเรียน (tag) ที่หน่วยงานนั้นถูก assign ให้จัดการ\\
2               & DATE\textunderscore{}ADDED            & เวลาที่หมวดหมู่เรื่องร้องเรียนนั้น ๆ ถูก assign ให้กับหน่วยงาน\\
\hline
\end{tabular}

\oaksimsec{ตัวอย่างข้อมูล}

\begin{lstlisting}[caption={\texttt{data/deps/human\textunderscore{}resources/tags.csv}}]
"TAG","DATE_ADDED"
"personnel","0"
\end{lstlisting}

\begin{description}[labelwidth=*]
    \item[$ \Rightarrow $] ในหน่วยงานที่มีชื่อว่า ``human\textunderscore{}resources'' มี ``personnel'' เป็นหนึ่งในหมวดหมู่เรื่องร้องเรียนใน\relax
\mbox{ความรับผิดชอบ} โดยถูก assign เมื่อเวลา 00:00:00 UTC ของวันพฤหัสบดีที่ 1 มกราคม ค.ศ. 1970
\end{description}
\end{minipage}


\vspace{3\baselineskip}


% F4
\begin{minipage}{\textwidth}
\subsubsection[\texttt{data/deps/deps.csv}]{\texorpdfstring{$ F_{4} $}{File \#4}: \texorpdfstring{\parbox[t]{0.5\textwidth}{\texttt{data/deps/\\deps.csv}}}{deps/deps.csv}}\label{subsubsec:csv-f4}
\phantomsection

ตารางของหน่วยงานผู้ใช้ Staff ทั้งหมดในระบบ

\begin{tabular}[!hbt]{| r | >{\ttfamily}p{15ex}<{\rmfamily} | p{36ex} |}
\hline
Column \#       & \rmfamily Column Title                & คำอธิบาย\\
\hline
1               & DEP\textunderscore{}ID                & หมายเลขประจำหน่วยงาน\\
2               & DEP\textunderscore{}NAME              & ชื่อของหน่วยงาน\\
3               & LEADER\textunderscore{}STAFF          & หมายเลขประจำตัวของผู้ใช้ Staff ซึ่งเป็นหัวหน้า (leader) ของหน่วยงาน หากมีค่าเป็น ``\texttt{NIL}'' หมายความว่าผู้ใช้ Admin ยังไม่ assign ผู้ใช้ Staff ให้เป็นหัวหน้าของหน่วยงาน\\
\hline
\end{tabular}

\oaksimsec{ตัวอย่างข้อมูล}

\begin{lstlisting}[caption={\texttt{data/deps/deps.csv}}]
"DEP_ID","DEP_NAME","LEADER_STAFF"
"9999999999999-department","human_resources","NIL"
\end{lstlisting}

\begin{description}[labelwidth=*]
    \item[$ \Rightarrow $] ในระบบมีหน่วยงานหมายเลข ``\texttt{9999999999999-department}'' ซึ่งมีชื่อว่า ``human\textunderscore{}resources'' อยู่ในระบบ
หน่วยงานนี้ยังไม่มีผู้ใช้ Staff เป็นหัวหน้า
\end{description}
\end{minipage}


\vspace{3\baselineskip}


% F5
\begin{minipage}{\textwidth}
\subsubsection[\texttt{data/issues/complaints/evidences/evidences.csv}]{\texorpdfstring{$ F_{5} $}{File \#5}: \texorpdfstring{\parbox[t]{0.5\textwidth}{\texttt{data/issues/\\complaints/evidences/evidences.csv}}}{issues/complaints/evidences/evidences.csv}}\label{subsubsec:csv-f5}
\phantomsection

ตารางของหลักฐานแนบเรื่องร้องเรียนทั้งหมดในระบบ

\begin{tabular}[!hbt]{| r | >{\ttfamily}p{15ex}<{\rmfamily} | p{36ex} |}
\hline
Column \#       & \rmfamily Column Title                & คำอธิบาย\\
\hline
1               & EVIDENCE\textunderscore{}ID           & หมายเลขประจำหลักฐานแนบ (Evidence ID)\\
2               & EVIDENCE\textunderscore{}PATH         & relative path จาก \texttt{data/issues/complaints/evidences/} ไปสู่ไฟล์หลักฐานแนบ ---  \marginnote{เมื่อผู้ใช้ Consumer ได้แนบไฟล์หลักฐานเรื่องร้องเรียน ไฟล์ดังกล่าวจะถูกทำสำเนาและจัดเก็บใน directory ภายในแอป โดยสำเนาของไฟล์จะถูกตั้งชื่อเป็น \texttt{<COMPLAINT\textunderscore{}ID>.<EXTENSION>} ยกตัวอย่างเช่น ``\texttt{5555555555555-complaint.jpeg}''} ซึ่งก็คือชื่อของไฟล์ไฟล์หลักฐานแนบนั่นเอง\\
\hline
\end{tabular}

\oaksimsec{ตัวอย่างข้อมูล}

\begin{lstlisting}[caption={\texttt{data/issues/complaints/evidences/evidences.csv}}]
"EVIDENCE_ID","EVIDENCE_PATH"
"5555555555555-evidence","5555555555555-complaint.jpeg"
\end{lstlisting}

\begin{description}[labelwidth=*]
    \item[$ \Rightarrow $] ในระบบมีหลักฐานแนบหมายเลข ``\texttt{5555555555555-evidence}''
สามารถเข้าถึงได้ด้วยการ resolve ``\texttt{data/issues/complaints/evidences/}'' กับ ``\texttt{5555555555555-complaint.jpeg}''
\end{description}
\end{minipage}


\vspace{3\baselineskip}


% F6
\begin{minipage}{\textwidth}
\subsubsection[\texttt{data/issues/complaints/categories.csv}]{\texorpdfstring{$ F_{6} $}{File \#6}: \texorpdfstring{\parbox[t]{0.5\textwidth}{\texttt{data/issues/\\complaints/categories.csv}}}{issues/complaints/categories.csv}}\label{subsubsec:csv-f6}
\phantomsection

ตารางของหมวดหมู่เรื่องร้องเรียนทั้งหมดในระบบ

\begin{tabular}[!hbt]{| r | >{\ttfamily}p{15ex}<{\rmfamily} | p{36ex} |}
\hline
Column \#       & \rmfamily Column Title                & คำอธิบาย\\
\hline
1               & CATEGORY                              & หมวดหมู่เรื่องร้องเรียน (Complaint Category)\\
2               & DATE\textunderscore{}ADDED            & เวลาที่เพิ่มหมวดหมู่เรื่องร้องเรียนเข้าสู่ระบบ\\
\hline
\end{tabular}

\oaksimsec{ตัวอย่างข้อมูล}

\begin{lstlisting}[caption={\texttt{data/issues/complaints/categories.csv}}]
"CATEGORY","DATE_ADDED"
"personnel","0"
\end{lstlisting}

\begin{description}[labelwidth=*]
    \item[$ \Rightarrow $] ในระบบมีหมวดหมู่เรื่องร้องเรียน ``personnel''
โดยถูกเพิ่มเข้ามาเมื่อเวลา 00:00:00 UTC ของวันพฤหัสบดีที่ 1 มกราคม ค.ศ. 1970
\end{description}
\end{minipage}


\vspace{3\baselineskip}


\newpage

% F7
\begin{minipage}{\textwidth}
\subsubsection[\texttt{data/issues/complaints/complaints.csv}]{\texorpdfstring{$ F_{7} $}{File \#7}: \texorpdfstring{\parbox[t]{0.5\textwidth}{\texttt{data/issues/\\complaints/complaints.csv}}}{issues/complaints/complaints.csv}}\label{subsubsec:csv-f7}
\phantomsection

ตารางของเรื่องร้องเรียนทั้งหมดในระบบ

\begin{tabular}[!hbt]{| r | >{\ttfamily}p{18ex}<{\rmfamily} | p{33ex} |}
\hline
Column \#       & \rmfamily Column Title                & คำอธิบาย\\
\hline
1               & COMPLAINT\textunderscore{}ID          & หมายเลขประจำเรื่องร้องเรียน\\
2               & AUTHOR\textunderscore{}UID            & หมายเลขประจำตัวผู้ใช้ของผู้ร้องเรียน\\
3               & CATEGORY                              & หมวดหมู่เรื่องร้องเรียน\\
4               & SUBJECT                               & หัวเรื่องของเรื่องร้องเรียน\\
5               & DESCRIPTION                           & คำอธิบายของเรื่องร้องเรียน\\
6               & EVIDENCE\textunderscore{}PATH         & relative path จาก \texttt{data/issues/complaints/evidences/} ไปสู่ไฟล์หลักฐาน\\
7               & VOTERS                                & colon-separated list ของหมายเลขประจำตัวผู้ใช้ซึ่งลงคะแนนเสียงให้กับเรื่องร้องเรียน\\
8               & STATUS                                & สถานะปัจจุบันของเรื่องร้องเรียน\\
9               & CASE\textunderscore{}MANAGER\textunderscore{}UID  & หมายเลขประจำตัวผู้ใช้ของผู้ใช้ Staff ซึ่งเป็นผู้รับผิดชอบเรื่องร้องเรียน หากมีค่าเป็น ``\texttt{NIL}'' หมายความว่า เรื่องร้องเรียนยังไม่ถูกจัดการโดยผู้ใช้ Staff คนใดในระบบ --- หรืออีกนัยหนึ่ง (imply) คือ เรื่องร้องเรียนอยู่ในสถานะ ``รอจัดการ'' (Pending)\\
\hline
\end{tabular}
\end{minipage}

\clearpage

\begin{minipage}{\textwidth}
\oaksimsec{ตัวอย่างข้อมูล}

\begin{lstlisting}[caption={\texttt{data/issues/complaints/complaints.csv}},breakatwhitespace=false,breaklines=false]
%\rmfamily(\#1) %"COMPLAINT_ID",%\lstsuppressnumbers%
%\rmfamily(\#2) %"AUTHOR_UID",
%\rmfamily(\#3) %"CATEGORY",
%\rmfamily(\#4) %"SUBJECT",
%\rmfamily(\#5) %"DESCRIPTION",
%\rmfamily(\#6) %"EVIDENCE_PATH",
%\rmfamily(\#7) %"VOTERS",
%\rmfamily(\#8) %"STATUS",
%\rmfamily(\#9) %"CASE_MANAGER_UID"%\lstrestorenumbers{2}%
%\rmfamily(\#1) %"5555555555555-complaint",%\lstsuppressnumbers%
%\rmfamily(\#2) %"2000000000000-consumeruser",
%\rmfamily(\#3) %"personnel",
%\rmfamily(\#4) %"Teaching Too Good",
%\rmfamily(\#5) %"I'm going to describe this ...",
%\rmfamily(\#6) %"NIL",
%\rmfamily(\#7) %"2000000000001-consumeruser:2000000000002-consumeruser",
%\rmfamily(\#8) %"PENDING",
%\rmfamily(\#9) %"NIL"%\lstrestorenumbers{1}%
\end{lstlisting}

\begin{description}[labelwidth=*]
    \item[$ \Rightarrow $] (\#1) ในระบบมีเรื่องร้องเรียนหมายเลข ``\texttt{5555555555555-complaint}''\\
(\#2) ซึ่งแจ้งโดยผู้ใช้หมายเลข ``\texttt{2000000000000-consumeruser}''\\
(\#3) เป็นเรื่องร้องเรียนในหมวดหมู่ ``personnel''\\
(\#4) มีหัวเรื่องว่า ``Teaching Too Good''\\
(\#5) ผู้แจ้งได้แจงรายละเอียดว่า ``I'm going to describe this ...''\\
(\#6) เรื่องร้องเรียนนี้ไม่มีหลักฐานแนบ\\
(\#7) มีผู้ใช้หมายเลข ``\texttt{2000000000001-consumeruser}''\\
\hspace{4ex}และ ``\texttt{2000000000002-consumeruser}'' ลงคะแนนเสียงให้\\
(\#8) เรื่องร้องเรียนนี้มีสถานะ ``รอการจัดการ''\\
(\#9) ยังไม่มีผู้ใช้ Staff เป็นผู้รับผิดชอบ
\end{description}
\end{minipage}

\clearpage

\vspace{3\baselineskip}


% F8
\begin{minipage}{\textwidth}
\subsubsection[\texttt{data/issues/reports/reports.csv}]{\texorpdfstring{$ F_{8} $}{File \#8}: \texorpdfstring{\parbox[t]{0.5\textwidth}{\texttt{data/issues/\\reports/reports.csv}}}{issues/reports/reports.csv}}\label{subsubsec:csv-f8}
\phantomsection

ตารางคำรายงานความไม่เหมาะสมทั้งหมดในระบบ

\begin{tabular}[!hbt]{| r | >{\ttfamily}p{15ex}<{\rmfamily} | p{36ex} |}
\hline
Column \#       & \rmfamily Column Title                & คำอธิบาย\\
\hline
1               & REPORT\textunderscore{}ID             & หมายเลขคำรายงาน\\
2               & TYPE                                  & ประเภทคำรายงาน ประกอบด้วย \mbox{`พฤติกรรมไม่เหมาะสม'} (\texttt{BEHAVIOR}) และ \mbox{`เนื้อหาไม่เหมาะสม'} (\texttt{CONTENT}) \\
3               & AUTHOR\textunderscore{}ID             & หมายเลขประจำตัวของผู้ใช้ผู้รายงาน\\
4               & TARGET\textunderscore{}ID             & หมายเลขประจำตัวของผู้ใช้ผู้ถูกรายงาน\\
5               & DESCRIPTION                           & คำอธิบายของคำรายงาน\\
6               & STATUS                                & สถานะปัจจุบันของคำรายงาน\\
7               & RESULT                                & ผลการพิจารณาคำรายงาน\\
\hline
\end{tabular}

\oaksimsec{ตัวอย่างข้อมูล}

\begin{lstlisting}[caption={\texttt{data/issues/reports/reports.csv}},breakatwhitespace=false,breaklines=false]
%\rmfamily(\#1) %"REPORT_ID",%\lstsuppressnumbers%
%\rmfamily(\#2) %"TYPE",
%\rmfamily(\#3) %"AUTHOR_UID",
%\rmfamily(\#4) %"TARGET_ID",
%\rmfamily(\#5) %"DESCRIPTION",
%\rmfamily(\#6) %"STATUS",
%\rmfamily(\#7) %"RESULT"%\lstrestorenumbers{2}%
%\rmfamily(\#1) %"6666666666666-report",%\lstsuppressnumbers%
%\rmfamily(\#2) %"BEHAVIOR",
%\rmfamily(\#3) %"2000000000000-consumeruser",
%\rmfamily(\#4) %"2000000000001-consumeruser",
%\rmfamily(\#5) %"Spamming complaints",
%\rmfamily(\#6) %"PENDING",
%\rmfamily(\#7) %"NIL"%\lstrestorenumbers{1}%
\end{lstlisting}

\begin{description}[labelwidth=*]
    \item[$ \Rightarrow $] (\#1) คำรายงานหมายเลข ``\texttt{6666666666666-report}''\\
(\#2) เป็นคำรายงานประเภท ``พฤติกรรมไม่เหมาะสม''\\
(\#3) ผู้รายงานคือผู้ใช้หมายเลข ``\texttt{2000000000000-consumeruser}''\\
(\#4) ผู้ถูกรายงานคือผู้ใช้หมายเลข ``\texttt{2000000000001-consumeruser}''\\
(\#5) คำอธิบายของคำรายงานคือ ``Spamming complaints''\\
(\#6) สถานะของคำรายงานคือ ``รอการจัดการ''\\
(\#7) ยังไม่ทราบผลการพิจารณา
\end{description}
\end{minipage}


\vspace{3\baselineskip}


% F9
\begin{minipage}{\textwidth}
\subsubsection[\texttt{data/users/USER/info.csv}]{\texorpdfstring{$ F_{9} $}{File \#9}: \texorpdfstring{\parbox[t]{0.5\textwidth}{\texttt{data/users/\\USER/info.csv}}}{users/USER/info.csv}}\label{subsubsec:csv-f9}
\phantomsection

ข้อมูลทั่วไปของผู้ใช้

\begin{tabular}[!hbt]{| r | >{\ttfamily}p{18ex}<{\rmfamily} | p{33ex} |}
\hline
Column \#       & \rmfamily Column Title                & คำอธิบาย\\
\hline
1               & UID                                   & หมายเลขประจำตัวผู้ใช้\\
2               & USER\textunderscore{}NAME             & ชื่อผู้ใช้\\
3               & ROLE                                  & ระดับสิทธิ์ของผู้ใช้\\
4               & FIRST\textunderscore{}NAME            & ชื่อต้น/ชื่อจริง ของผู้ใช้\\
5               & LAST\textunderscore{}NAME             & นามสกุลของผู้ใช้\\
6               & USING\textunderscore{}DEFAULT \textunderscore{}PROFILE\textunderscore{}IMAGE      & ใช้รูป profile เป็นรูปตั้งต้นของระบบใช่หรือไม่ --- \texttt{true} หรือ \texttt{false} ตามลำดับ\\
7               & PROFILE\textunderscore{}IMAGE \textunderscore{}EXT                                & นามสกุลของไฟล์รูป profile ซึ่งจัดเก็บอยู่ในระบบ\\
\hline
\end{tabular}

\oaksimsec{ตัวอย่างข้อมูล}

\begin{lstlisting}[caption={\texttt{data/users/jdoe/info.csv}}]
%\rmfamily(\#1) %"UID",%\lstsuppressnumbers%
%\rmfamily(\#2) %"USER_NAME",
%\rmfamily(\#3) %"ROLE",
%\rmfamily(\#4) %"FIRST_NAME",
%\rmfamily(\#5) %"LAST_NAME",
%\rmfamily(\#6) %"USING_DEFAULT_PROFILE_IMAGE",
%\rmfamily(\#7) %"PROFILE_IMAGE_EXT"%\lstrestorenumbers{2}%
%\rmfamily(\#1) %"2000000000001-consumeruser",%\lstsuppressnumbers%
%\rmfamily(\#2) %"jdoe",
%\rmfamily(\#3) %"CONSUMER",
%\rmfamily(\#4) %"John",
%\rmfamily(\#5) %"Doe",
%\rmfamily(\#6) %"true",
%\rmfamily(\#7) %"png"%\lstrestorenumbers{1}%
\end{lstlisting}

\begin{description}[labelwidth=*]
    \item[$ \Rightarrow $] (\#1) ผู้ใช้หมายเลข ``\texttt{2000000000001-consumeruser}''\\
(\#2) มีชื่อผู้ใช้ ``jdoe''\\
(\#3) เป็นผู้ใช้ที่มีระดับสิทธิ์ ``Consumer''\\
(\#4) มีชื่อต้น/ชื่อจริง ``John''\\
(\#5) มีนามสกุล ``Doe''\\
(\#6) ใช้รูปตั้งต้นของระบบเป็นรูป profile ของผู้ใช้\\
(\#7) นามสกุลของไฟล์รูป profile คือ ``\texttt{png}''
\end{description}
\end{minipage}


\vspace{3\baselineskip}


% F10
\begin{minipage}{\textwidth}
\subsubsection[\texttt{data/users/requests.csv}]{\texorpdfstring{$ F_{10} $}{File \#10}: \texorpdfstring{\parbox[t]{0.5\textwidth}{\texttt{data/users/requests.csv}}}{users/requests.csv}}\label{subsubsec:csv-f10}
\phantomsection

ตารางของคำขอปลดระงับสิทธิ์การใช้งานของผู้ใช้ Consumer ต่อผู้ใช้ Admin

\begin{tabular}[!hbt]{| r | >{\ttfamily}p{15ex}<{\rmfamily} | p{36ex} |}
\hline
Column \#       & \rmfamily Column Title                & คำอธิบาย\\
\hline
1               & UID                                   & หมายเลขประจำตัวผู้ใช้\\
2               & REPORT\textunderscore{}ID             & หมายเลขคำรายงานซึ่งเป็นเหตุให้ผู้ใช้ถูกระงับสิทธิ์การใช้งาน\\
3               & MESSAGE                               & เหตุผลของการขอปลดระงับสิทธิ์การใช้งาน\\
\hline
\end{tabular}

\oaksimsec{ตัวอย่างข้อมูล}

\begin{lstlisting}[caption={\texttt{data/users/requests.csv}}]
"UID","REPORT_ID","MESSAGE"
"2000000000001-consumeruser","6666666666666-report", "I don't know why my account was suspended. Unsuspend me NOW or get yourself a new car."
\end{lstlisting}

\begin{description}[labelwidth=*]
    \item[$ \Rightarrow $] คำขอปลดระงับสิทธิ์การใช้งานจากผู้ใช้หมายเลข ``\texttt{2000000000001-consumeruser}''
ซึ่งถูกระงับสิทธิ์โดยมีคำรายงานหมายเลข ``\texttt{6666666666666-report}'' เป็นต้นเหตุ
ผู้ใช้ยังได้แนบเหตุผลการขอปลดระงับสิทธิ์การใช้งานมาด้วยว่า ``I don't know why my account was suspended. Unsuspend me NOW or get yourself a new car.''
\end{description}
\end{minipage}


\vspace{3\baselineskip}


% F11
\begin{minipage}{\textwidth}
\subsubsection[\texttt{data/passwd.csv}]{\texorpdfstring{$ F_{11} $}{File \#11}: \texttt{data/passwd.csv}}\label{subsubsec:csv-f11}
\phantomsection

ตารางของรหัสผ่านสำหรับลงชื่อเข้าสู่ระบบของผู้ใช้ในระบบ

\begin{tabular}[!hbt]{| r | >{\ttfamily}p{15ex}<{\rmfamily} | p{36ex} |}
\hline
Column \#       & \rmfamily Column Title                & คำอธิบาย\\
\hline
1               & UID                                   & หมายเลขประจำตัวผู้ใช้\\
2               & PASSWORD                              & รหัสผ่าน\\
\hline
\end{tabular}

\oaksimsec{ตัวอย่างข้อมูล}

\begin{lstlisting}[caption={\texttt{data/passwd.csv}}]
"UID","PASSWORD"
"2000000000001-consumeruser","im_a_spammer"
\end{lstlisting}

\begin{description}[labelwidth=*]
    \item[$ \Rightarrow $] ผู้ใช้หมายเลข ``\texttt{2000000000001-consumeruser}''
ได้ตั้งรหัสผ่านสำหรับลงชื่อเข้าสู่ระบบเป็น ``\texttt{im\textunderscore{}a\textunderscore{}spammer}''
\end{description}
\end{minipage}


\vspace{3\baselineskip}


% F12
\begin{minipage}{\textwidth}
\subsubsection[\texttt{data/sessions.csv}]{\texorpdfstring{$ F_{12} $}{File \#12}: \texttt{data/sessions.csv}}\label{subsubsec:csv-f12}
\phantomsection

ตารางบันทึกเวลาการลงชื่อเข้าสู่ระบบครั้งล่าสุดของผู้ใช้ทุกคนในระบบ

\begin{tabular}[!hbt]{| r | >{\ttfamily}p{15ex}<{\rmfamily} | p{36ex} |}
\hline
Column \#       & \rmfamily Column Title                & คำอธิบาย\\
\hline
1               & ๊ID                                    & หมายเลขประจำตัวผู้ใช้\\
2               & TIME                                  & เวลาการลงชื่อเข้าสู่ระบบครั้งล่าสุดของผู้ใช้เป็น UNIX timestamp\\
\hline
\end{tabular}

\oaksimsec{ตัวอย่างข้อมูล}

\begin{lstlisting}[caption={\texttt{data/sessions.csv}}]
"UID","TIME"
"2000000000000-consumeruser","2147483647000"
\end{lstlisting}

\begin{description}[labelwidth=*]
    \item[$ \Rightarrow $] ผู้ใช้หมายเลข ``\texttt{2000000000000-consumeruser}'' ได้ลงชื่อเข้าสู่ระบบล่าสุดเมื่อ
เวลา 03:14:07 UTC ของวันอังคารที่ 19 มกราคม ค.ศ. 2038
\end{description}
\end{minipage}


\vspace{3\baselineskip}


% F13
\begin{minipage}{\textwidth}
\subsubsection[\texttt{data/users.csv}]{\texorpdfstring{$ F_{13} $}{File \#13}: \texttt{data/users.csv}}\label{subsubsec:csv-f13}
\phantomsection

ตารางของผู้ใช้ทั้งหมดในระบบ

\begin{tabular}[!hbt]{| r | >{\ttfamily}p{15ex}<{\rmfamily} | p{36ex} |}
\hline
Column \#       & \rmfamily Column Title                & คำอธิบาย\\
\hline
1               & UID                                   & หมายเลขประจำตัวผู้ใช้\\
2               & USER\textunderscore{}NAME             & ชื่อผู้ใช้\\
3               & ROLE                                  & ระดับสิทธิ์ของผู้ใช้\\
4               & IS\textunderscore{}ACTIVE             & ผู้ใช้ไม่ได้ถูกระงับสิทธิ์อยู่ใช่หรือไม่ --- ``\texttt{true}'' หรือ ``\texttt{false}'' ตามลำดับ\\
5               & LOGIN\textunderscore{}ATTEMPT         & จำนวนครั้งในการพยายามลงชื่อเข้าสู่ระบบในขณะที่ถูกระงับสิทธิ์อยู่\\
\hline
\end{tabular}

\oaksimsec{ตัวอย่างข้อมูล}

\begin{lstlisting}[caption={\texttt{data/users.csv}}]
%\rmfamily(\#1) %"UID",%\lstsuppressnumbers%
%\rmfamily(\#2) %"USER_NAME",
%\rmfamily(\#3) %"ROLE",
%\rmfamily(\#4) %"IS_ACTIVE",
%\rmfamily(\#5) %"LOGIN_ATTEMPT"%\lstrestorenumbers{2}%
%\rmfamily(\#1) %"2000000000001-consumeruser",%\lstsuppressnumbers%
%\rmfamily(\#2) %"jdoe",
%\rmfamily(\#3) %"CONSUMER",
%\rmfamily(\#4) %"false",
%\rmfamily(\#5) %"1418211"%\lstrestorenumbers{1}%
\end{lstlisting}

\begin{description}[labelwidth=*]
    \item[$ \Rightarrow $] (\#1) ผู้ใช้หมายเลข ``\texttt{2000000000001-consumeruser}''\\
(\#2) มีชื่อผู้ใช้ ``jdoe''\\
(\#3) เป็นผู้ใช้ที่มีระดับสิทธิ์ ``Consumer''\\
(\#4) ปัจจุบันถูกระงับสิทธิ์การใช้งานอยู่\\
(\#5) ได้พยายามลงชื่อ\mbox{เข้าสู่ระบบ}ในขณะที่ถูกระงับสิทธิ์อยู่ นับจนถึงปัจจุบัน\\
\hspace{4ex}รวม ``1418211'' ครั้ง
\end{description}
\end{minipage}
