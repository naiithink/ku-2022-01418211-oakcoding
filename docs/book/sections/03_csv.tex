\section{รายละเอียดไฟล์ CSV}

\subsection{การตีความไฟล์ CSV ใน \href{https://github.com/CS211-651/project211-oakcoding/tree/9397d355461933fb007261e2ee97445ea93eacc1/data}{\texttt{data\slash}~Directory}}

ไฟล์ CSV ใน \href{https://github.com/CS211-651/project211-oakcoding/tree/9397d355461933fb007261e2ee97445ea93eacc1/data}{\texttt{data\slash}~directory} จะมีรูปแบบร่วมกันคือ
ในแต่ละไฟล์จะมีหัวตาราง (header) และมีส่วนที่เป็นเนื้อหาของไฟล์ (body)

ยกตัวอย่างเช่น ข้อมูลในตาราง people ต่อไปนี้:

\begin{table}[!htb]
\begin{tabular}{@{} | l | l | l | l | l | @{}}
\hline
USER\textunderscore{}ID     & NAME       & EMAIL\\
\hline\hline
0001        & Bob        & bob@example.com\\
0002        & Alice      & alice@example.com\\
\hline
\end{tabular}
\caption{people}\label{tab:people}
\end{table}

สามารถเก็บเป็นไฟล์ CSV ได้เป็น

\begin{lstlisting}[caption={\texttt{people.csv}},label={lst:people.csv}]
"USER_ID","NAME","EMAIL"
"0001","Bob","bob@example.com"
"0002","Alice","alice@example.com"
\end{lstlisting}

header\marginpar{header ของตารางของไฟล์ CSV ใน project จะอยู่ที่บรรทัดแรกของไฟล์เสมอและจะถูกเขียนด้วยตัวอักษรภาษาอังกฤษพิมพ์ใหญ่ (uppercase) ทั้งหมดโดยที่ไม่มีตัวอักษรว่าง (space)} ของตารางมีเอาไว้เพื่อกำหนดว่าข้อมูลใน column/field นั้น ๆ คืออะไร
นั่นก็เพื่ออำนวยให้ทั้งคนและแอปสามารถตีความข้อมูลได้

ส่วน body ของตารางมีเอาไว้เพื่อเก็บข้อมูล โดยที่ข้อมูล 1 entity จะกินพื้นที่ 1 row/record ในตาราง
และจะวางตัวเป็น tuple; รูปแบบการ represent ข้อมูลอย่างมีลำดับ อย่างเช่น $ (x, y) $ จะเป็นคนละ entity กับ $ (y, x) $

และเมื่อต้องตีความข้อมูลจากไฟล์ \nameref{lst:people.csv} ก็สามารถตีความได้ดังนี้

\begin{enumerate}
    \item ไฟล์ที่มีชื่อว่า \texttt{people.csv} เป็นไฟล์ CSV ที่มี 3 columns
    \item แต่ละ column ในไฟล์แสดงข้อมูล \texttt{UID}, \texttt{NAME}, และ \texttt{EMAIL} ตามลำดับ
    \item ในไฟล์นั้นมี 1 record ที่ represent ข้อมูล:\marginpar{เริ่มตั้งแต่บรรทัดที่ 2 ของไฟล์ ลำดับของบรรทัดนั้นไม่มีนัยสำคัญ}
        \begin{itemize}
            \item \texttt{UID}: 0001
            \item \texttt{NAME}: Bob
            \item \texttt{EMAIL}: bob@example.com
        \end{itemize}
    \item และในไฟล์เดียวกันก็มีอีก 1 record ที่ represent ข้อมูล:
        \begin{itemize}
            \item \texttt{UID}: 0002
            \item \texttt{NAME}: Alice
            \item \texttt{EMAIL}: alice@example.com
        \end{itemize}
\end{enumerate}

คุณสามารถนำวิธีการตีความไฟล์ CSV นี้ไปปรับใช้กับไฟล์ CSV ใด ๆ ก็ได้ใน \texttt{data/} directory ของ project

\clearpage

\subsection{ไฟล์ CSV ใน \href{https://github.com/CS211-651/project211-oakcoding/tree/9397d355461933fb007261e2ee97445ea93eacc1/data}{\texttt{data\slash}~Directory}}

นี่คือโครงสร้างคร่าว ๆ เพื่อแสดงที่อยู่ของไฟล์ CSV ใน data/ directory

\begin{lstlisting}[name={ไฟล์ CSV ใน data/ Directory},numbers=none]
data/
    ...
    deps/
        *<DEPARTMENT>/
            info.csv -------------------------------% \hyperref[subsubsec:csv-f1]{$ F_{1} $} %
            members.csv ----------------------------% \hyperref[subsubsec:csv-f2]{$ F_{2} $} %
            tags.csv -------------------------------% \hyperref[subsubsec:csv-f3]{$ F_{3} $} %
        ...
        deps.csv -----------------------------------% \hyperref[subsubsec:csv-f4]{$ F_{4} $} %
    issues/
        complaints/
            evidences/
                evidences.csv ----------------------% \hyperref[subsubsec:csv-f5]{$ F_{5} $} %
            categories.csv -------------------------% \hyperref[subsubsec:csv-f6]{$ F_{6} $} %
            complaints.csv -------------------------% \hyperref[subsubsec:csv-f7]{$ F_{7} $} %
        reports/
            reports.csv ----------------------------% \hyperref[subsubsec:csv-f8]{$ F_{8} $} %
    users/
        *<USER>/
            info.csv -------------------------------% \hyperref[subsubsec:csv-f9]{$ F_{9} $} %
        ...
        requests.csv -------------------------------% \hyperref[subsubsec:csv-f10]{$ F_{10} $} %
    passwd.csv -------------------------------------% \hyperref[subsubsec:csv-f11]{$ F_{11} $} %
    sessions.csv -----------------------------------% \hyperref[subsubsec:csv-f12]{$ F_{12} $} %
    users.csv --------------------------------------% \hyperref[subsubsec:csv-f13]{$ F_{13} $} %
\end{lstlisting}

% UNDERSCORED VERSION
% \begin{lstlisting}[name={ไฟล์ CSV ใน data/ Directory},numbers=none]
% data/
%     ...
%     deps/
%         *<DEPARTMENT>/
%             info.csv _______________________________% \hyperref[subsubsec:csv-f1]{$ F_{1} $} %
%             members.csv ____________________________% \hyperref[subsubsec:csv-f2]{$ F_{2} $} %
%             tags.csv _______________________________% \hyperref[subsubsec:csv-f3]{$ F_{3} $} %
%         ...
%         deps.csv ___________________________________% \hyperref[subsubsec:csv-f4]{$ F_{4} $} %
%     issues/
%         complaints/
%             evidences/
%                 evidences.csv ______________________% \hyperref[subsubsec:csv-f5]{$ F_{5} $} %
%             categories.csv _________________________% \hyperref[subsubsec:csv-f6]{$ F_{6} $} %
%             complaints.csv _________________________% \hyperref[subsubsec:csv-f7]{$ F_{7} $} %
%         reports/
%             reports.csv ____________________________% \hyperref[subsubsec:csv-f8]{$ F_{8} $} %
%     users/
%         *<USER>/
%             info.csv _______________________________% \hyperref[subsubsec:csv-f9]{$ F_{9} $} %
%         ...
%         requests.csv _______________________________% \hyperref[subsubsec:csv-f10]{$ F_{10} $} %
%     passwd.csv _____________________________________% \hyperref[subsubsec:csv-f11]{$ F_{11} $} %
%     sessions.csv ___________________________________% \hyperref[subsubsec:csv-f12]{$ F_{12} $} %
%     users.csv ______________________________________% \hyperref[subsubsec:csv-f13]{$ F_{13} $} %
% \end{lstlisting}

และเพื่อความสะดวกในการอ้างอิง เราจะให้ $ F_n $ แทนไฟล์ \texttt{*.csv} ที่ $ n $

\clearpage

\subsection{รายละเอียดของไฟล์ CSV แต่ละไฟล์}


% F1
\begin{minipage}{\textwidth}
\subsubsection{\texorpdfstring{$ F_{1} $}{File \#1}: \texttt{data/deps/DEPARTMENT/info.csv}}\label{subsubsec:csv-f1}
\phantomsection

ข้อมูลทั่วไปของ department

\begin{tabular}[!hbt]{| r | >{\ttfamily}p{15ex}<{\rmfamily} | p{36ex} |}
\hline
Column \#       & \rmfamily Column Title                & คำอธิบาย\\
\hline
1               & DEP\textunderscore{}ID                & หมายเลขประจำ department\\
2               & DEF\textunderscore{}NAME              & ชื่อของ department\\
\hline
\end{tabular}

\oaksimsec{ตัวอย่างข้อมูล}

\begin{lstlisting}[caption={\texttt{data/deps/human\textunderscore{}resources/info.csv}}]
"DEP_ID","DEP_NAME"
"0000000000001-department","human_resources"
\end{lstlisting}

\begin{description}[labelwidth=*]
    \item[$ \Rightarrow $]  Department ที่มี ID ``\texttt{0000000000001-department}'' เป็น department ที่มีชื่อว่า ``human\textunderscore{}resources''
\end{description}
\end{minipage}


% F2
\begin{minipage}{\textwidth}
\subsubsection{\texorpdfstring{$ F_{2} $}{File \#2}: \texttt{**/DEPARTMENT/members.csv}}\label{subsubsec:csv-f2}
\phantomsection

รายชื่อของ staff member ใน department

\begin{tabular}[!hbt]{| r | >{\ttfamily}p{15ex}<{\rmfamily} | p{36ex} |}
\hline
Column \#       & \rmfamily Column Title                & คำอธิบาย\\
\hline
1               & UID                                   & หมายเลขประจำตัวผู้ใช้ของ staff ที่เป็น member ของ department \\
2               & DATE\textunderscore{}ADDED            & เวลาที่เพิ่ม member คนนั้น ๆ เข้าสู่ department\\
\hline
\end{tabular}

\oaksimsec{ตัวอย่างข้อมูล}

\begin{lstlisting}[caption={\texttt{data/deps/human\textunderscore{}resources/members.csv}}]
"UID","DATE_ADDED"
"0000000000001-staffuser","0000000000000"
\end{lstlisting}

\begin{description}[labelwidth=*]
    \item[$ \Rightarrow $]  ใน department ที่มีชื่อว่า ``human\textunderscore{}resources'' มีผู้ใช้ member อยู่ 1 ผู้ใช้ที่มี
\mbox{หมายเลขประจำตัวผู้ใช้} ``\texttt{0000000000001-staffuser}'' โดยถูกเพิ่มเข้ามา\relax
เมื่อเวลา 00:00:00 UTC ของวันพฤหัสบดีที่ 1 มกราคม ค.ศ. 1970
\end{description}
\end{minipage}


% F3
\begin{minipage}{\textwidth}
\subsubsection{\texorpdfstring{$ F_{3} $}{File \#3}: \texttt{**/DEPARTMENT/members.csv}}\label{subsubsec:csv-f3}
\phantomsection

ข้อมูลทั่วไปของ department

\begin{tabular}[!hbt]{| r | >{\ttfamily}p{15ex}<{\rmfamily} | p{36ex} |}
\hline
Column \#       & \rmfamily Column Title                & คำอธิบาย\\
\hline
1               & DEP\textunderscore{}ID                & หมายเลขประจำ department\\
2               & DATE\textunderscore{}ADDED            & เวลาที่เพิ่ม member คนนั้น ๆ เข้าสู่ department\\
\hline
\end{tabular}

\oaksimsec{ตัวอย่างข้อมูล}

\begin{lstlisting}
"UID","DATE_ADDED"
"0000000000000-adminuser",""
\end{lstlisting}
\end{minipage}


% F4
\begin{minipage}{\textwidth}
\subsubsection{\texorpdfstring{$ F_{4} $}{File \#4}: \texttt{**/DEPARTMENT/members.csv}}\label{subsubsec:csv-f4}
\phantomsection

ข้อมูลทั่วไปของ department

\begin{tabular}[!hbt]{| r | >{\ttfamily}p{15ex}<{\rmfamily} | p{36ex} |}
\hline
Column \#       & \rmfamily Column Title                & คำอธิบาย\\
\hline
1               & DEP\textunderscore{}ID                & หมายเลขประจำ department\\
2               & DATE\textunderscore{}ADDED            & เวลาที่เพิ่ม member คนนั้น ๆ เข้าสู่ department\\
\hline
\end{tabular}

\oaksimsec{ตัวอย่างข้อมูล}

\begin{lstlisting}
"UID","DATE_ADDED"
"0000000000000-adminuser",""
\end{lstlisting}
\end{minipage}


% F5
\begin{minipage}{\textwidth}
\subsubsection{\texorpdfstring{$ F_{5} $}{File \#5}: \texttt{**/DEPARTMENT/members.csv}}\label{subsubsec:csv-f5}
\phantomsection

ข้อมูลทั่วไปของ department

\begin{tabular}[!hbt]{| r | >{\ttfamily}p{15ex}<{\rmfamily} | p{36ex} |}
\hline
Column \#       & \rmfamily Column Title                & คำอธิบาย\\
\hline
1               & DEP\textunderscore{}ID                & หมายเลขประจำ department\\
2               & DATE\textunderscore{}ADDED            & เวลาที่เพิ่ม member คนนั้น ๆ เข้าสู่ department\\
\hline
\end{tabular}

\oaksimsec{ตัวอย่างข้อมูล}

\begin{lstlisting}
"UID","DATE_ADDED"
"0000000000000-adminuser",""
\end{lstlisting}
\end{minipage}


% F6
\begin{minipage}{\textwidth}
\subsubsection{\texorpdfstring{$ F_{6} $}{File \#6}: \texttt{**/DEPARTMENT/members.csv}}\label{subsubsec:csv-f6}
\phantomsection

ข้อมูลทั่วไปของ department

\begin{tabular}[!hbt]{| r | >{\ttfamily}p{15ex}<{\rmfamily} | p{36ex} |}
\hline
Column \#       & \rmfamily Column Title                & คำอธิบาย\\
\hline
1               & DEP\textunderscore{}ID                & หมายเลขประจำ department\\
2               & DATE\textunderscore{}ADDED            & เวลาที่เพิ่ม member คนนั้น ๆ เข้าสู่ department\\
\hline
\end{tabular}

\oaksimsec{ตัวอย่างข้อมูล}

\begin{lstlisting}
"UID","DATE_ADDED"
"0000000000000-adminuser",""
\end{lstlisting}
\end{minipage}


% F7
\begin{minipage}{\textwidth}
\subsubsection{\texorpdfstring{$ F_{7} $}{File \#7}: \texttt{**/DEPARTMENT/members.csv}}\label{subsubsec:csv-f7}
\phantomsection

ข้อมูลทั่วไปของ department

\begin{tabular}[!hbt]{| r | >{\ttfamily}p{15ex}<{\rmfamily} | p{36ex} |}
\hline
Column \#       & \rmfamily Column Title                & คำอธิบาย\\
\hline
1               & DEP\textunderscore{}ID                & หมายเลขประจำ department\\
2               & DATE\textunderscore{}ADDED            & เวลาที่เพิ่ม member คนนั้น ๆ เข้าสู่ department\\
\hline
\end{tabular}

\oaksimsec{ตัวอย่างข้อมูล}

\begin{lstlisting}
"UID","DATE_ADDED"
"0000000000000-adminuser",""
\end{lstlisting}
\end{minipage}


% F8
\begin{minipage}{\textwidth}
\subsubsection{\texorpdfstring{$ F_{8} $}{File \#8}: \texttt{**/DEPARTMENT/members.csv}}\label{subsubsec:csv-f8}
\phantomsection

ข้อมูลทั่วไปของ department

\begin{tabular}[!hbt]{| r | >{\ttfamily}p{15ex}<{\rmfamily} | p{36ex} |}
\hline
Column \#       & \rmfamily Column Title                & คำอธิบาย\\
\hline
1               & DEP\textunderscore{}ID                & หมายเลขประจำ department\\
2               & DATE\textunderscore{}ADDED            & เวลาที่เพิ่ม member คนนั้น ๆ เข้าสู่ department\\
\hline
\end{tabular}

\oaksimsec{ตัวอย่างข้อมูล}

\begin{lstlisting}
"UID","DATE_ADDED"
"0000000000000-adminuser",""
\end{lstlisting}
\end{minipage}


% F9
\begin{minipage}{\textwidth}
\subsubsection{\texorpdfstring{$ F_{9} $}{File \#9}: \texttt{**/DEPARTMENT/members.csv}}\label{subsubsec:csv-f9}
\phantomsection

ข้อมูลทั่วไปของ department

\begin{tabular}[!hbt]{| r | >{\ttfamily}p{15ex}<{\rmfamily} | p{36ex} |}
\hline
Column \#       & \rmfamily Column Title                & คำอธิบาย\\
\hline
1               & DEP\textunderscore{}ID                & หมายเลขประจำ department\\
2               & DATE\textunderscore{}ADDED            & เวลาที่เพิ่ม member คนนั้น ๆ เข้าสู่ department\\
\hline
\end{tabular}

\oaksimsec{ตัวอย่างข้อมูล}

\begin{lstlisting}
"UID","DATE_ADDED"
"0000000000000-adminuser",""
\end{lstlisting}
\end{minipage}


% F10
\begin{minipage}{\textwidth}
\subsubsection{\texorpdfstring{$ F_{10} $}{File \#10}: \texttt{**/DEPARTMENT/members.csv}}\label{subsubsec:csv-f10}
\phantomsection

ข้อมูลทั่วไปของ department

\begin{tabular}[!hbt]{| r | >{\ttfamily}p{15ex}<{\rmfamily} | p{36ex} |}
\hline
Column \#       & \rmfamily Column Title                & คำอธิบาย\\
\hline
1               & DEP\textunderscore{}ID                & หมายเลขประจำ department\\
2               & DATE\textunderscore{}ADDED            & เวลาที่เพิ่ม member คนนั้น ๆ เข้าสู่ department\\
\hline
\end{tabular}

\oaksimsec{ตัวอย่างข้อมูล}

\begin{lstlisting}
"UID","DATE_ADDED"
"0000000000000-adminuser",""
\end{lstlisting}
\end{minipage}


% F11
\begin{minipage}{\textwidth}
\subsubsection{\texorpdfstring{$ F_{11} $}{File \#11}: \texttt{**/DEPARTMENT/members.csv}}\label{subsubsec:csv-f11}
\phantomsection

ข้อมูลทั่วไปของ department

\begin{tabular}[!hbt]{| r | >{\ttfamily}p{15ex}<{\rmfamily} | p{36ex} |}
\hline
Column \#       & \rmfamily Column Title                & คำอธิบาย\\
\hline
1               & DEP\textunderscore{}ID                & หมายเลขประจำ department\\
2               & DATE\textunderscore{}ADDED            & เวลาที่เพิ่ม member คนนั้น ๆ เข้าสู่ department\\
\hline
\end{tabular}

\oaksimsec{ตัวอย่างข้อมูล}

\begin{lstlisting}
"UID","DATE_ADDED"
"0000000000000-adminuser",""
\end{lstlisting}
\end{minipage}


% F12
\begin{minipage}{\textwidth}
\subsubsection{\texorpdfstring{$ F_{12} $}{File \#12}: \texttt{**/DEPARTMENT/members.csv}}\label{subsubsec:csv-f12}
\phantomsection

ข้อมูลทั่วไปของ department

\begin{tabular}[!hbt]{| r | >{\ttfamily}p{15ex}<{\rmfamily} | p{36ex} |}
\hline
Column \#       & \rmfamily Column Title                & คำอธิบาย\\
\hline
1               & DEP\textunderscore{}ID                & หมายเลขประจำ department\\
2               & DATE\textunderscore{}ADDED            & เวลาที่เพิ่ม member คนนั้น ๆ เข้าสู่ department\\
\hline
\end{tabular}

\oaksimsec{ตัวอย่างข้อมูล}

\begin{lstlisting}
"UID","DATE_ADDED"
"0000000000000-adminuser",""
\end{lstlisting}
\end{minipage}


% F13
\begin{minipage}{\textwidth}
\subsubsection{\texorpdfstring{$ F_{13} $}{File \#13}: \texttt{**/DEPARTMENT/members.csv}}\label{subsubsec:csv-f13}
\phantomsection

ข้อมูลทั่วไปของ department

\begin{tabular}[!hbt]{| r | >{\ttfamily}p{15ex}<{\rmfamily} | p{36ex} |}
\hline
Column \#       & \rmfamily Column Title                & คำอธิบาย\\
\hline
1               & DEP\textunderscore{}ID                & หมายเลขประจำ department\\
2               & DATE\textunderscore{}ADDED            & เวลาที่เพิ่ม member คนนั้น ๆ เข้าสู่ department\\
\hline
\end{tabular}

\oaksimsec{ตัวอย่างข้อมูล}

\begin{lstlisting}
"UID","DATE_ADDED"
"0000000000000-adminuser",""
\end{lstlisting}
\end{minipage}
