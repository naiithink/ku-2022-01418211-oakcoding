\section{รายละเอียดไฟล์ CSV}

\subsection{ไฟล์ CSV ใน \href{https://github.com/CS211-651/project211-oakcoding/tree/9397d355461933fb007261e2ee97445ea93eacc1/data}{\texttt{data\slash}~Directory}}

\begin{lstlisting}[name={ไฟล์ CSV ใน data/ Directory},numbers=none]
data/
    ...
    deps/
        <DEPARTMENTS' DIRECTORIES>/
            info.csv
            members.csv
            tags.csv
        ...
        deps.csv
    issues/
        complaints/
            evidences/
                evidences.csv
            categories.csv
            complaints.csv
        reports/
            reports.csv
    users/
        _ROOT/
            info.csv
        <USERS' DIRECTORIES>/
            info.csv
        ...
        requests.csv
    passwd.csv
    sessions.csv
    users.csv
\end{lstlisting}

\clearpage

\subsection{การตีความไฟล์ CSV ใน \href{https://github.com/CS211-651/project211-oakcoding/tree/9397d355461933fb007261e2ee97445ea93eacc1/data}{\texttt{data\slash}~Directory}}

ไฟล์ CSV ใน \href{https://github.com/CS211-651/project211-oakcoding/tree/9397d355461933fb007261e2ee97445ea93eacc1/data}{\texttt{data\slash}~directory} จะมีรูปแบบร่วมกันคือ
ในแต่ละไฟล์จะมีหัวตาราง (header) และมีส่วนที่เป็นเนื้อหาของไฟล์ (body)

ยกตัวอย่างเช่น ข้อมูลในตาราง people ต่อไปนี้:

\begin{table}[!htb]
\begin{tabular}{@{} | l | l | l | l | l | @{}}
\hline
USER\textunderscore{}ID     & NAME       & EMAIL\\
\hline\hline
0001        & Bob        & bob@example.com\\
0002        & Alice      & alice@example.com\\
\hline
\end{tabular}
\caption{people}\label{tab:people}
\end{table}

สามารถเก็บเป็นไฟล์ CSV ได้เป็น

\begin{lstlisting}[caption={\texttt{people.csv}},label={lst:people.csv},numbers=none]
"USER_ID","NAME","EMAIL"
"0001","Bob","bob@example.com"
"0002","Alice","alice@example.com"
\end{lstlisting}

header ของตารางมีเอาไว้เพื่อกำหนดว่าข้อมูลใน column/field นั้น ๆ คืออะไร
นั่นก็เพื่ออำนวยให้ทั้งคนและแอปสามารถตีความข้อมูลได้

ส่วน body ของตารางมีเอาไว้เพื่อเก็บข้อมูล โดยที่ข้อมูล 1 entity จะกินพื้นที่ 1 row/record ในตาราง
และจะวางตัวเป็น tuple; รูปแบบการ represent ข้อมูลอย่างมีลำดับ อย่างเช่น $ (x, y) $ จะเป็นคนละ entity กับ $ (y, x) $

และเมื่อต้องตีความข้อมูลจากไฟล์ \nameref{lst:people.csv} ก็สามารถตีความได้ดังนี้

\begin{enumerate}
    \item ไฟล์ที่มีชื่อว่า \texttt{people.csv} เป็นไฟล์ CSV ที่มี 3 columns
    \item แต่ละ column ในไฟล์เก็บข้อมูล \texttt{USER\textunderscore{}ID}, \texttt{NAME}, \texttt{EMAIL} ตามลำดับ
    \item Record ที่ 1 ของไฟล์นั้น represent ข้อมูล:
        \begin{itemize}
            \item \texttt{USER\textunderscore{}ID}: 0001
            \item \texttt{NAME}: Bob
            \item \texttt{EMAIL}: bob@example.com
        \end{itemize}
    \item Record ที่ 2 ของไฟล์นั้น represent ข้อมูล:
        \begin{itemize}
            \item \texttt{USER\textunderscore{}ID}: 0002
            \item \texttt{NAME}: Alice
            \item \texttt{EMAIL}: alice@example.com
        \end{itemize}
\end{enumerate}

คุณสามารถนำวิธีการตีความไฟล์ CSV นี้ไปปรับใช้กับไฟล์ CSV ใด ๆ ก็ได้ใน \texttt{data/} directory ของ project

\subsection{รายละเอียดของไฟล์ CSV แต่ละไฟล์}
\noindent\blindtext[3]
