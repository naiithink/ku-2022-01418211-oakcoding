% !TEX program = xelatex
\documentclass{article}


\usepackage{oakcodingdocs}


%% Sections
\setcounter{section}{0}


%% Cover Page
\title{%
\fontsize{32pt}{\baselineskip}\flushleft\textbf{\docsname}\\
\vspace{1ex}
\fontsize{22pt}{\baselineskip}\ourTeam\\
\vspace{3ex}
\large Desktop Application สำหรับการแจ้งเรื่องร้องเรียนของนิสิตมหาวิทยาลัยเกษตรศาสตร์}
\author{}
\date{}


\setlength{\baselineskip}{3ex}


%% Document Body
\begin{document}
\begin{titlepage}
\newgeometry{
includemp = false}
\maketitle
\thispagestyle{empty}
\vspace{1ex}

% \noindent\rule{\paperwidth - \oddsidemargin}{0.6pt}

\rule{0em}{2ex}

\LARGE \textbf{คำชี้แจงที่สำคัญ (Disclaimer)} \normalsize

\rule{0em}{3ex}

\large \textbf{ทีมผู้พัฒนา OakCoding} \normalsize

\begin{enumerate}
    \setlength{\itemsep}{0.7pt}
    \item ปาณชัย คชกาษร
    \item ธนากร คนหมั่น
    \item ธเนศ จีนสีคง
    \item พศวัต ถิ่นกาญจน์วัฒนา
\end{enumerate}

\rule{0em}{3ex}

\large \textbf{ระบบปฏิบัติการที่รองรับ} \normalsize\\
OakCoding เป็น desktop application ที่ถูกพัฒนาและรองรับการใช้งานบนระบบปฏิบัติการ Apple macOS\\
ทีมผู้พัฒนา OakCoding ไม่รับประกันว่า OakCoding จะสามารถทำงานบนระบบปฏิบัติการอื่นใดได้อย่างสมบูรณ์

\rule{0em}{3ex}

\large \textbf{ข้อจำกัดความรับผิดชอบ} \normalsize\\
OakCoding ถูกพัฒนาขึ้นเพื่อจุดประสงค์ทางการศึกษา\\
ทีมผู้พัฒนา OakCoding จะไม่รับผิดชอบต่อความเสียหายใด ๆ รวมถึง ความเสียหายทางตรง ความเสียหายทางอ้อม\\
ความเสียหายโดยบังเอิญ หรือความเสียหายเกี่ยวเนื่อง อันเป็นผลหรือสืบเนื่องจากการใช้งาน OakCoding
\end{titlepage}

\clearpage

\newgeometry{
includemp = false}
\thispagestyle{empty}
\tableofcontents
\restoregeometry
\newpage

\subfile{sections/00_introduction.tex}

\clearpage

\subfile{sections/01_usage}

\clearpage

\subfile{sections/02_seed}

\clearpage

\subfile{sections/03_csv}

\clearpage

\subfile{sections/04_extras}

\clearpage
\end{document}
